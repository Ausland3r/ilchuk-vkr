\chapter{Тестирование системы формирования CAPAs на основе изменений репозитория кода} \section{Введение}
В ходе тестирования проверяется работоспособность разработанного инструментария и выполняются поставленные в работе задачи. Основными целями данного этапа являются:
\begin{itemize}
	\item Проверка корректности функционирования пользовательского интерфейса и всех компонентов системы (сбора данных, анализа репозиториев, генерации рекомендаций).
	\item Оценка качества работы классификатора рисковых коммитов на реальных данных, полученных из активных репозиториев.
	\item Тестирование модели генерации CAPA (рекомендаций исправлений) для выявления ошибок и неточностей в алгоритме.
	\item Модульное и нагрузочное тестирование компонентов системы для выявления багов на уровне отдельных модулей и проверки масштабируемости на больших репозиториях.
\end{itemize}
Таким образом, экспериментальное тестирование охватывает проверку как функциональных характеристик (работа интерфейса, правильность алгоритмов), так и нефункциональных (надёжность, производительность) аспектов системы.

\section{Методика проведения тестирования}
Для проверки системы было выбрано несколько тестовых репозиториев с собственными разработками: \texttt{Tq}, \texttt{scherBook}, \texttt{NTO2024-2025}, а также репозитории предоставленные сторонними разработчиками и студентами, которые согласились поучаствовать в тестировании и апробации проекта: \texttt{urlagushka/polytech-labs}, \texttt{urlagushka/h8-pipeline}, \texttt{AlPakh/topotik-backend}, \texttt{AlPakh/topotik-frontend}, \texttt{Pacan4ik/tf-idf}, \texttt{Pacan4ik/tinkoff-course-spring2023}. Эти проекты содержат достаточно разнообразный код (на Java, Python, JavaScript, C++) и различную историю изменений, что обеспечивает репрезентативность данных.

\begin{longtable}{|p{3.5cm}|p{2.2cm}|p{2cm}|p{3cm}|p{4.5cm}|}
	\caption{Описание репозиториев, использованных в тестировании}
	\label{tab:test_repos} \\
	\hline
	\textbf{Репозиторий} & \textbf{Язык(и)} & \textbf{Кол-во коммитов} & \textbf{Тип проекта} & \textbf{Краткое описание} \\
	\hline
	\endfirsthead
	
	\multicolumn{5}{c}{{\tablename\ \thetable{}}} \\
	\hline
	\textbf{Репозиторий} & \textbf{Язык(и)} & \textbf{Кол-во коммитов} & \textbf{Тип проекта} & \textbf{Краткое описание} \\
	\hline
	\endhead
	
	\hline \multicolumn{5}{r}{{}} \\
	\endfoot
	
	\hline
	\endlastfoot
	
	\texttt{jup-ag/
		pyth-
		crosschain} & Python, Solidity & 3692 & Инфраструк-
		тура Web3 & Форк проекта Jup-ag с доработками Pyth network: кроссчейн модуль для верификации транзакций. \\
	\hline
	\texttt{Pacan4ik/
		tinkoff-course-
		spring2023} & Java & 188 & Курсовой проект & Проект на Java Spring Boot с интеграцией Telegram-бота и веб-интерфейсом. \\
	\hline
	\texttt{urlagushka/
		h8-pipeline} & Python & 110 & ML/AI пайплайн & Инициализатор проекта на базе Hailo SDK; используется для компьютерного зрения. \\
	\hline
	\texttt{AlPakh/
		topotik-
		backend} & Python (FastAPI) & 38 & Серверная часть & REST API бэкенд с авторизацией, ORM-моделями и обработкой карт. \\
	\hline
	\texttt{AlPakh/
		topotik-
		frontend} & JavaScript (Vue) & 27 & Клиентская часть & Vue-приложение с маршрутизацией, формами и подключением к API. \\
	\hline
	\texttt{Pacan4ik/
		tf-idf} & Python & 36 & Алгоритм обработки текста & Базовая реализация TF-IDF для анализа текстов на русском языке. \\
	\hline
	\texttt{Ausland3r/
		NTO2024-2025} & Python & 69 & Учебный проект & Материалы к проекту "Умный город" для школьников; используется в рамках NTO. \\
	\hline
	\texttt{urlagushka/
		polytech-labs} & C++, Java & 82 & Учебный проект & Набор лабораторных работ по программной инженерии; содержит решения на нескольких языках. \\
	\hline
	\texttt{Ausland3r/
		scherBook} & JavaScript & 39 & Веб-приложение & Клиентское приложение для платформы книгообмена; реализован основной UI. \\
	\hline
	\texttt{Ausland3r/
		Tq} & Python & 40 & Фреймворк для тестов & Небольшой собственный фреймворк на базе Pytest и Pydantic. \\
\end{longtable}

\section{Сравнение моделей классификации}

Для оценки эффективности алгоритмов машинного обучения в задаче предсказания рисковых коммитов был разработан универсальный класс \texttt{CommitRiskModel}. Он предоставляет единый интерфейс к различным классификаторам из \texttt{scikit-learn}, а также \texttt{deep-forest}. Класс реализует методы \texttt{fit()}, \texttt{predict()}, \texttt{predict\_proba()}, \texttt{feature\_importances()} и \texttt{evaluate\_model()}.

Для сравнения были обучены и протестированы десять моделей на выборках коммитов с автоматически определёнными метками риска. Эксперименты проводились как на небольших проектах (в пределах сотни коммитов), так и на объёмном репозитории \texttt{jup-ag/pyth-crosschain} (более 3000 коммитов).

\begin{table}[h!]
	\centering
	\caption{Сравнение моделей классификации на малом проекте (Pacan4ik/tinkoff-course-spring2023)}
	\label{tab:metrics_comparison_small_extended}
	\resizebox{\textwidth}{!}{
		\begin{tabular}{|l|c c c c|}
			\hline
			\textbf{Модель} & \textbf{Precision} & \textbf{Recall} & \textbf{F1-score} & \textbf{ROC-AUC} \\
			\hline
			LogisticRegression & 0.600 & 1.000 & 0.750 & 0.990 \\
			RandomForest       & 0.800 & 0.667 & 0.727 & 0.979 \\
			ExtraTrees         & 0.800 & 0.667 & 0.727 & 0.984 \\
			GradientBoosting   & 1.000 & 0.833 & 0.909 & 0.995 \\
			AdaBoost           & 1.000 & 0.833 & 0.909 & 0.979 \\
			XGBoost            & 0.800 & 0.667 & 0.727 & 0.979 \\
			LightGBM           & 0.800 & 0.667 & 0.727 & 0.958 \\
			CatBoost           & 0.833 & 0.833 & 0.833 & 0.990 \\
			SVM                & 0.833 & 0.833 & 0.833 & 0.990 \\
			DeepForest         & 0.833 & 0.833 & 0.833 & 0.990 \\
			\hline
		\end{tabular}
	}
\end{table}

На проекте Pacan4ik/tinkoff-course-spring2023 лучшие результаты по полноте (Recall = 0.833) и F1-score (0.909) показывают модели GradientBoosting и AdaBoost. Модели DeepForest, SVM и CatBoost демонстрируют сбалансированное сочетание метрик с F1 около 0.833 и высоким ROC-AUC (0.990), что свидетельствует о стабильной и надёжной работе. LogisticRegression достигает максимальной полноты (Recall = 1.0), но при этом точность ниже (Precision = 0.6), что говорит о склонности к избыточному помечанию коммитов как рисковых с увеличением числа ложных срабатываний. Модели RandomForest, ExtraTrees и XGBoost показывают одинаковые метрики, отражающие умеренный баланс между точностью и полнотой. В целом, DeepForest проявляет хорошую устойчивость при обучении на небольшом объёме данных.

\begin{table}[h!]
	\centering
	\caption{Сравнение моделей классификации на большом проекте}
	\label{tab:metrics_comparison_large_extended}
	\resizebox{\textwidth}{!}{
		\begin{tabular}{|l|c c c c|}
			\hline
			\textbf{Модель} & \textbf{Precision} & \textbf{Recall} & \textbf{F1-score} & \textbf{ROC-AUC} \\
			\hline
			LogisticRegression & 0.696 & 0.941 & 0.800 & 0.996 \\
			RandomForest       & 0.917 & 0.647 & 0.759 & 0.995 \\
			ExtraTrees         & 0.933 & 0.824 & 0.875 & 0.998 \\
			GradientBoosting   & 0.933 & 0.824 & 0.875 & 0.933 \\
			AdaBoost           & 0.875 & 0.824 & 0.848 & 0.952 \\
			XGBoost            & 0.867 & 0.765 & 0.812 & 0.962 \\
			LightGBM           & 0.818 & 0.529 & 0.643 & 0.946 \\
			CatBoost           & 0.867 & 0.765 & 0.812 & 0.994 \\
			SVM                & 0.938 & 0.882 & 0.909 & 0.999 \\
			DeepForest         & 0.842 & 0.941 & 0.889 & 0.996 \\
			\hline
		\end{tabular}
	}
\end{table}

На большом проекте \texttt{pyth-crosschain} наблюдается более разнообразная картина качества моделей. Модель SVM показывает наилучшие результаты по ключевым метрикам: высокая точность (0.938), хорошая полнота (0.882), сбалансированный F1-score (0.909) и практически идеальный ROC-AUC (0.999). Это говорит о том, что SVM эффективно выявляет рисковые коммиты, минимизируя как ложные срабатывания, так и пропуски.
Ансамблевые модели ExtraTrees и GradientBoosting также демонстрируют высокую полноту (Recall = 0.824) и хорошие показатели по остальным метрикам, показывая стабильность и надёжность на больших данных.
LogisticRegression и DeepForest выделяются очень высокой полнотой (около 0.94), что означает, что они почти не пропускают рисковые коммиты. Однако у DeepForest точность немного ниже, вероятно из-за особенности архитектуры каскадных слоёв, которая может быть чувствительна к шуму в данных, что приводит к большему количеству ложных срабатываний.
В целом, SVM показывает наилучшее сочетание точности и полноты, что делает эту модель предпочтительным выбором для задач классификации рисковых коммитов в больших репозиториях.

Таким образом, можно выделить следующие рекомендации:
\begin{itemize}
	\item Для небольших проектов с ограниченной историей коммитов предпочтительнее использовать \texttt{DeepForest} и ансамблевые методы, чтобы минимизировать пропущенные риски и достичь сбалансированных метрик.
	\item Для крупных репозиториев рекомендуется \texttt{SVM} и методы градиентного бустинга \texttt{XGBoost} \texttt{CatBoost} как наиболее устойчивые и точные модели.
\end{itemize}

В итоге можно с уверенностью сказать, что разработанный унифицированный класс \texttt{CommitRiskModel} позволяет гибко подключать и тестировать разные алгоритмы без необходимости переписывать код, что делает систему легко масштабируемой и адаптируемой под различные сценарии и объёмы данных.

\section{Модульное тестирование}

Для обеспечения качества и надёжности разработанного программного обеспечения выполнено модульное тестирование ключевых компонентов системы. В рамках тестирования было создано и выполнено 12 отдельных тестов, охватывающих основные сценарии работы и критичные граничные случаи. Основные направления тестирования и проверяемые аспекты включают:

\begin{itemize}
	\item \texttt{ml\_model.py}:
Проверялись основные методы обучения и предсказания моделей машинного обучения. Тесты, такие как \verb|test_model_extremes_and_balanced_cases| и \verb|test_model_predict_|
\verb|proba_output|, гарантировали корректную работу метода \texttt{fit()}, проверяли способность модели обучаться на реальных и частично неполных данных, а также корректно обрабатывать ситуации с отсутствием или неполнотой входных данных. Валидация предсказаний включала проверку возвращаемых значений — скалярных вероятностей в диапазоне от 0 до 1.

Данный набор тестов помогает своевременно обнаруживать ошибки, связанные с изменениями в логике обучения и предсказания, предотвращая ошибки при работе с моделями машинного обучения.

\item \texttt{repository\_analysis.py}:

Проверялся процесс сбора и обработки статистики по коммитам из репозитория. В тестах, например, \texttt{test\_repository\_empty\_and\_corrupted} были учтены разные ситуации: работа с нормальным репозиторием с несколькими коммитами, а также с пустым репозиторием или с некорректными данными или отсутствующими файлами. Тесты гарантировали, что модуль не выдаёт ошибок при отсутствии данных, а возвращает корректные пустые структуры.  

Это снижает риск сбоев в работе системы при взаимодействии с нестандартными или пустыми репозиториями.

\item \texttt{recommendations.py}: 

Проверялись функции генерации рекомендаций на основе анализа изменений в коммитах. Тест \texttt{test\_recommendations\_extreme\_commit} моделирует ситуации с коммитами без добавленных строк кода, наличием уже существующих рекомендаций. Проверялась корректность формируемых списков рекомендаций.

Это позволяет гарантировать, что рекомендации будут релевантными и не содержат дублирующей или ошибочной информации.

\item \texttt{app.py}:

Этот модуль содержит ключевые функции, обеспечивающие загрузку и анализ данных из репозиториев, обучение модели машинного обучения и обновление табличного представления результатов. Тесты \texttt{test\_load\_and\_analyze\_repos}, \texttt{test\_train\_and\_update\_model} и \texttt{test\_update\_tabs} проверяют корректность выполнения основных операций.
\end{itemize}

Все тесты были автоматизированы с использованием фреймворка \texttt{pytest} и обеспечили покрытие ключевых функциональных частей системы более чем на \verb|75%|. Такой уровень тестового покрытия подтверждает надёжность и стабильность реализации, а также значительно упрощает сопровождение и дальнейшее развитие кода, обеспечивая своевременное выявление регрессий и ошибок.

\begin{figure}[ht]
	\centering
	\includegraphics[width=\textwidth]{my_folder/images/test_result.png}
	\caption{Отчёт с результатом прогона автотестов в pytest-html}
	\label{tab:test_pytest}
\end{figure}

\begin{figure}[ht]
	\centering
	\includegraphics[width=\textwidth]{my_folder/images/coverage.png}
	\caption{Отчёт по покрытию автотестами в pytest-cov}
	\label{tab:test_pytest2}
\end{figure}


\section{Опыт написания мутационных тестов}

В дополнение к стандартному функциональному тестированию, для повышения качества проверки реализован набор мутационных тестов. Мутационные тесты направлены на выявление слабых мест в валидации и логике обработки данных модели \texttt{CommitRiskModel} и системы рекомендаций, проверяя реакцию системы на некорректные или граничные данные.

В процессе разработки мутационных тестов были выявлены следующие типичные ошибки и соответствующие решения:

\begin{itemize}
	\item \texttt{Отсутствие проверки обязательных полей в данных коммитов.} При попытке подачи данных с отсутствующими критичными признаками (например, \verb|avg_file_history| или \verb|message_length|) модель изначально не выбрасывала ошибку. Для устранения этой проблемы в класс \texttt{CommitRiskModel} была добавлена функция \verb|_validate_commits|, которая строго проверяет наличие всех обязательных полей и типы данных, а также их корректность (например, запрет отрицательных значений). Теперь такие ситуации корректно вызывают исключения \verb|KeyError| или \verb|ValueError|, что усиливает устойчивость к ошибкам данных.
	\item \texttt{Отсутствие валидации корректности значений признаков.} Выявлены случаи, когда модель не корректно обрабатывала отрицательные или неверного типа значения (например, отрицательное число добавленных строк или строка вместо числа). Исправлено добавлением строгой типизации и проверки значений, что повышает качество входных данных и уменьшает риск неожиданного поведения.
	
	\item \texttt{Проверка генерации рекомендаций при специфических условиях.} Были обнаружены случаи, когда рекомендации по высоким рискам не содержали ожидаемых ключевых фраз (например, "углублённое код-ревью"), либо предупреждения при низком риске неожиданно появлялись. Для устранения этих несоответствий было принято решение дополнительно выводить отладочную информацию в тестах, а также расширить проверки, сделав их более гибкими по формулировкам и ключевым словам. Это позволило избежать ложных срабатываний и улучшить интерпретируемость рекомендаций.
	
	\item \texttt{Обработка нестандартных сценариев работы метода \texttt{predict\_proba}}. В мутационных тестах искусственно подменялся метод \texttt{predict\_proba}, возвращающий некорректные вероятности вне диапазона [0,1]. После этого тесты выявляли неправильное поведение. Такая проверка показала необходимость гарантировать валидность вероятностных выходов в производственном коде или строго проверять результаты в тестах.
	

\section{Нагрузочное тестирование}

Нагрузочное тестирование проводилось с целью оценки производительности системы при работе с крупными репозиториями. В качестве тестового примера был выбран репозиторий \texttt{jup-ag/pyth-crosschain} с более чем 3000 коммитов. Тестирование выполнялось на машине со следующими характеристиками: процессор AMD Ryzen 9 7900X, 16 ГБ оперативной памяти и SSD-накопитель.

\begin{itemize}
	\item Общее время обработки полного репозитория составило 46 минут. Основная часть времени затрачивается на последовательный запуск статического анализа (Pylint, Checkstyle) для сотен файлов. Дополнительное время уходит на загрузку коммитов через API GitHub, особенно при большом количестве коммитов в репозитории.
	
	\item Наиболее ресурсоёмкой подсистемой оказался статический анализ: последовательный запуск линтеров на большом количестве файлов существенно увеличивает нагрузку на ресурсы компьютера. Максимальная загрузка CPU достигала 53\%, а использование оперативной памяти — до 8 ГБ.
\end{itemize}
Для ускорения работы возможно применение кеширования результатов анализа и параллельной обработки файлов. Для избежания повторной загрузки данных с GitHub система сохраняет локальную копию репозитория. Поскольку API GitHub имеет ограничения по скорости и количеству запросов, локальный клон позволяет минимизировать обращения к API и работать с полной историей и файлами непосредственно на диске, что ускоряет повторный анализ уже загруженных репозиториев. Также чтобы не было необходимости постоянно запускать систему для получения рекомендаций весь список рекомендаций сохраняется локально в md файл и отправляется в отдельную ветку в удаленном репозитории.


В целом система стабильно справлялась с обработкой крупного репозитория, обеспечивая корректные результаты и бесперебойную работу. Максимальная нагрузка приходилась на этап анализа качества кода. Полученные результаты подтверждают, что разработанные компоненты способны эффективно работать с реальными проектами значительных размеров.

\begin{figure}[ht]
	\centering
	\includegraphics[width=\textwidth]{my_folder/images/nagruzka.png}
	\caption{Запущенная система с репозиторием \texttt{jup-ag/pyth-crosschain}}
	\label{tab:nagruzka}
\end{figure}

\section{Обзор и интерпретация результатов}
Визуализация результатов анализа коммитов позволяет быстро выявлять тенденции в проекте. Рассмотрим пример репозитория \texttt{Ausland3r/NTO2024-2025}. На рисунках приведены разные аспекты анализа:

Рисунок~\ref{fig:commit_stats} показывает гистограммы базовых метрик коммитов. Видно, что большинство коммитов содержит меньше 10 добавленных или удалённых строк, а также влияет не более чем на 5 файлов. Сложность изменений (четвёртый график) в большинстве случаев небольшая. Такие диаграммы позволяют визуально оценить, что значительная часть коммитов малых по размеру и сложности, что характерно для аккуратного ведения проекта. 

 \begin{figure}[H]
	\centering
	\includegraphics[width=\textwidth]{my_folder/images/first_page.png}
	\caption{Распределение изменений в коммитах: добавленные и удалённые строки, количество изменённых файлов и сложность изменений для проекта \texttt{Ausland3r/NTO2024-2025}.}
	\label{fig:commit_stats}
\end{figure}
На Рисунке~\ref{fig:risk_analysis} представлена информация об эффективности классификатора на этом репозитории. В таблице видим Precision=0.40, Recall=1.00, что соответствует метрикам модели на этих данных. Круговая диаграмма показывает, что около \verb|11.8%| коммитов отмечены как рисковые (красным цветом). Справа виден график «Risk vs Complexity»: наблюдается тенденция, что коммиты с большей сложностью имеют более высокий риск (жёлтым - рисковые коммиты). Данный анализ помогает подтвердить, что алгоритм верно выделяет несколько потенциально проблемных коммитов (в основном с большой сложностью), и диаграммы наглядно демонстрируют распределение рисковых изменений.

\begin{figure}[H]
	\centering
	\includegraphics[width=\textwidth]{my_folder/images/second_page.png}
	\caption{Метрики классификации и распределение рисковых коммитов для репозитория \texttt{Ausland3r/NTO2024-2025}. Таблица показывает качество модели (Precision, Recall, F1, ROC-AUC), диаграмма слева — долю рисковых коммитов (красным), справа — зависимость риска от сложности.}
	\label{fig:risk_analysis}
\end{figure}

Рисунок~\ref{fig:authors} иллюстрирует вклад разных разработчиков. Слева видно, что основной объём коммитов внесли \texttt{Ausland3r} и \texttt{DenisovDmitrii} (по ~30 коммитов каждый), остальные авторы — единичные вклады. Справа график показывает средний риск по автору: например, \texttt{Fliegende\_Rehe} (средний риск ~0.4) выделяется как относительный «рисковый» автор, хотя у него было меньше коммитов. Такая визуализация помогает в определении, кто из участников при текущем анализе вносит больше потенциально проблемных изменений.

 \begin{figure}[ht]
	\centering
	\includegraphics[width=\textwidth]{my_folder/images/third_page.png}
	\caption{Активность авторов и средний риск на автора для проекта \texttt{Ausland3r/NTO2024-2025}. Слева — число коммитов на автора, справа — усреднённый риск.}
	\label{fig:authors}
\end{figure}
 \begin{figure}[ht]
	\centering
	\includegraphics[width=\textwidth]{my_folder/images/forth_page.png}
	\caption{Файловая карта риска (частота изменений vs средний риск) для \texttt{Ausland3r/NTO2024-2025}. Точка \texttt{Task/taskJ.py} выделена как файл с 8 изменениями и средним риском 0.25.}
	\label{fig:file_risk}
\end{figure}

На Рисунке~\ref{fig:file_risk} представлена файловая карта: по горизонтали — число изменений файла (\texttt{change\_count}), по вертикали — средний риск изменений этого файла. Замечено, что файл \texttt{Task/task1.py} менялся 8 раз и имеет средний риск ~0.38 (отмечен на графике). Большинство же файлов имеют низкий риск. 

Такая диаграмма позволяет выявлять «горячие точки» проекта — файлы, часто изменяющиеся и с высоким риском, требующие внимания. 


Наконец, Рисунок~\ref{fig:timeline} демонстрирует динамику проекта. По синей линии видно, что средний риск коммитов постепенно рос с ноября 2024 по декабрь 2024. В мае 2025 на проекте снова были внесены изменения. Оранжевые столбцы отображают количество предупреждений статического анализа во времени. Видно несколько пиков предупреждений в начале проекта; далее они стабилизировались на низком уровне. Такая временная диаграмма подчёркивает, как со временем изменялась стабильность проекта, и позволяет своевременно заметить всплески риска или предупреждений. В целом приведённые визуализации показывают, что проект велся относительно аккуратно. 



\begin{figure}[ht]
	\centering
	\includegraphics[width=\textwidth]{my_folder/images/fifth_page.png}
	\caption{Временная шкала риска и предупреждений для \texttt{Ausland3r/NTO2024-2025}. Синяя линия — средний риск коммитов с течением времени, оранжевые столбцы — число предупреждений статического анализа.}
	\label{fig:timeline}
\end{figure}

Визуальные представления, такие как вышеперечисленные, существенно упрощают анализ состояния проекта: они помогают быстро выделить проблемные области (рисковые коммиты, ответственные авторы, модифицируемые файлы и т.д.), которые трудно заметить при простом чтении логов. Аналитическая панель с такими графиками позволяет команде разработки эффективнее контролировать качество кода и прогнозировать потенциальные риски. 

\subsection{Описание страницы рекомендаций по коммитам}

На данной странице отображаются рекомендации по каждому коммиту репозитория, сгенерированные на основе анализа риска и метрик изменений. Рекомендации призваны помочь разработчикам и ревьюерам быстро оценить потенциальные проблемы и принять соответствующие меры.

\begin{figure}[ht]
	\centering
	\includegraphics[width=\textwidth]{my_folder/images/capa_page.png}
	\caption{Таблица рисков и предупреждений для \texttt{Ausland3r/NTO2024-2025}.}
	\label{fig:timeline123}
\end{figure}

Основные виды рекомендаций включают:

\begin{itemize}
	\item Высокий риск (например, вероятность риска выше 0.8): 
	\begin{itemize}
		\item Рекомендуется углублённое код-ревью и расширенное тестирование.
		\item При наличии — показаны наиболее значимые признаки, повлиявшие на оценку риска, чтобы понять причины высокой оценки.
	\end{itemize}
	\item Повышенный риск (риск в диапазоне 0.5–0.8): 
	\begin{itemize}
		\item Совет обратить внимание на качество изменений и добавить тесты.
	\end{itemize}
	\item Качество сообщения коммита:
	\begin{itemize}
		\item Сообщения с длиной менее 15 символов получают рекомендацию расширить описание изменений.
		\item Очень длинные сообщения (более 200 символов) предлагается структурировать или сократить.
	\end{itemize}
	\item Объём изменений:
	\begin{itemize}
		\item Если количество изменённых строк значительно превышает среднее по репозиторию (свыше среднего плюс два стандартных отклонения), рекомендуется разбивать изменения на более мелкие логические части.
	\end{itemize}
	\item Специфические сигналы:
	\begin{itemize}
		\item Коммиты с ключевыми словами, указывающими на исправление багов, сопровождаются рекомендацией проверить наличие регрессионных тестов и обновление документации.
	\end{itemize}
\end{itemize}

Реализация рекомендаций основана на анализе различных метрик коммита, таких как длина сообщения, объём изменений, количество изменённых файлов, а также вероятности риска, вычисленные моделью. Это позволяет делать выводы не только на основе простых пороговых значений, но и учитывать специфику репозитория (через статистики по проекту) и значимость отдельных признаков риска.

Таким образом, представленные рекомендации реально соответствуют конкретному коммиту и его характеристикам, помогая в раннем выявлении потенциальных проблем и улучшении процесса код-ревью.

\vspace{0.5em}
Пример рекомендаций по коммиту:

\begin{quote}
	\begin{itemize}
		\item Очень высокий риск: обязательно провести углублённое код-ревью и расширенное тестирование.
		\item  Наибольшее влияние на риск оказали признаки: lines\_deleted, pylint\_warnings, message\_length.
		\item  Сообщение слишком короткое: дайте подробное описание изменений.
		\item Объём изменений (150) значительно превышает среднее (50.3). Разбейте коммит на более мелкие логические части.
	\end{itemize}
\end{quote}

Такая система рекомендаций повышает прозрачность оценки качества коммитов и способствует улучшению практик разработки в команде.

\section{Сравнение экспертных меток и рекомендаций, выданных моделью}

Для оценки качества работы системы по формированию рекомендаций и анализа рисковых коммитов были использованы экспертные метки, предоставленные опытными разработчиками из компаний \texttt{СТЦ} и \texttt{Coşkunöz Engineering and Tehnological Solutions}. Целью данного этапа тестирования было сравнить рекомендации, полученные с помощью модели \texttt{CommitRiskModel}, с теми, которые были предложены экспертами, и оценить, насколько хорошо модель воспроизводит их опыт и рекомендации.

Эксперты дали ряд рекомендаций по анализу сообщений коммитов и рисков, связанным с ними, в то время как модель предоставила свои собственные предложения на основе метрик, извлечённых из анализа репозиториев. В этом разделе будет проведено сравнение их выводов, а также проанализирована эффективность работы модели по сравнению с экспертными оценками.

\subsection{Рекомендации экспертов}

Экспертами были выделены несколько типов проблем, характерных для коммитов, с подробными рекомендациями по их улучшению:

\begin{itemize}
	\item Merge-коммиты: Эксперты отметили, что слияние веток через \texttt{merge} приводит к неудобствам при анализе истории изменений. Рекомендуется использовать методы слияния через \texttt{fast-forward} или \texttt{squash}, что улучшает читаемость истории коммитов и облегчает создание более понятных сообщений для merge-коммитов.
	\item Не-ASCII символы в сообщениях: Сообщения, содержащие не-ASCII символы, затрудняют использование стандартных инструментов, таких как \texttt{grep}, и могут привести к ошибкам при генерации чейнджлогов. Рекомендуется придерживаться единообразия в использовании символов, избегая нестандартных символов в сообщениях.
	\item Неатомарные изменения: Эксперты подчеркнули, что коммиты, которые затрагивают несколько логически несвязанных изменений, затрудняют понимание истории репозитория. Рекомендуется разделять такие изменения на несколько атомарных коммитов.
	\item Неинформативные сообщения: Сообщения, которые не раскрывают суть изменения, могут создать проблемы в будущем при анализе истории изменений. Рекомендуется предоставлять более подробные и описательные сообщения, которые точно объясняют изменения, внесённые в коммите.
	\item Невалидный автор коммита: Если автор и коммитер не совпадают, это может свидетельствовать о плохой организации работы с репозиторием и нарушении норм командной работы. Эксперты советуют избегать таких ситуаций, корректируя данные о коммитере.
	\item Наличие бинарных файлов в коммитах: Бинарные файлы, такие как \texttt{.zip}, \texttt{.pickle} и другие, могут создать проблемы при использовании системы контроля версий, поскольку их невозможно эффективно сравнивать или сливать. Эксперты рекомендуют избегать добавления бинарников в репозитории.
	\item Невыполнение стиля сообщений коммитов: Сообщения коммитов должны быть структурированы в соответствии с общепринятыми стандартами (например, \texttt{Conventional Commits}). Отсутствие контекста в сообщении коммита затрудняет точное понимание изменений и приводит к ухудшению документации в репозитории.
	\item Проверка наличия тестов: Эксперты рекомендуют обязательно добавлять тесты, если изменения касаются функционала.
	\item Частота коммитов: Частое сохранение прогресса важно и полезно, однако избыточные коммиты могут указывать на неаккуратную работу или спешку, что может привести к ухудшению качества кода и усложнить ревью. Эксперты рекомендуют делать менее частые, но более осмысленные и качественные коммиты.
\end{itemize}

\subsection{Рекомендации модели}

Модель \texttt{CommitRiskModel} предлагает рекомендации на основе анализа ряда факторов, таких как длина сообщения коммита, объём изменений, количество изменённых файлов, а также вероятность риска, вычисленная моделью. Некоторые ключевые рекомендации, выданные моделью, включают:

\begin{itemize}
	\item Сообщение коммита слишком короткое: если сообщение содержит менее 15 символов, рекомендуется предоставить более подробное описание.
	\item Объём изменений превышает среднее: если изменения значительно превышают средний объём для репозитория, модель предлагает разделить коммит на более мелкие логические части.
	\item Ключевые слова в сообщении: если коммит содержит ключевые слова, такие как "fix" или "bug", рекомендуется проверить наличие регрессионных тестов.
	\item Merge-коммиты: модель рекомендует использовать методы слияния \texttt{squash} или \texttt{fast-forward}, а также составить нормальные сообщения для merge.
	\item Неатомарные изменения: модель предлагает разделить такие изменения на несколько более мелких коммитов для улучшения читаемости.
\end{itemize}

Пример рекомендации от модели:

Очень высокий риск: обязательно провести углублённое код-ревью и расширенное тестирование. Наибольшее влияние на риск оказали признаки: \verb|lines_deleted|, \verb|pylint_warnings|, \verb|message_length|. Сообщение слишком короткое: дайте подробное описание изменений. Объём изменений (150) значительно превышает среднее (50.3). Разбейте коммит на более мелкие логические части.

\subsection{Сравнение рекомендаций экспертов и модели}

Для более детального анализа была составлена таблица, сравнивающая рекомендации, предложенные экспертами, с теми, что были выданы моделью:

\begin{table}[h!]
	\centering
	\caption{Сравнение рекомендаций экспертов и модели для репозитория Pacan4ik/tf-idf}
	\label{tab:expert_vs_model_recommendations}
	\small
	\begin{tabularx}{\textwidth}{|p{2.5cm}|X|X|}
		\hline
		\textbf{SHA коммита} & \textbf{Рекомендации эксперта} & \textbf{Рекомендации модели} \\
		\hline
		6aa0ac9, f9dba1c & Merge-коммиты лучше делать с использованием стандартных сообщений (например, squash или fast-forward) для удобства чтения истории. & Merge-коммиты, неудобно для анализа, рекомендуется использовать fast-forward или squash \\
		\hline
		ca6655a, 079146f & Сообщения содержат не-ASCII символы, улучшите читаемость & В сообщении обнаружены нестандартные символы. Желательно использовать только ASCII \\
		\hline
		6ad7d36, 838f481 & Неатомарные изменения, разделите на логически завершённые части & Неатомарные изменения, разделите на более мелкие, логически завершённые коммиты \\
		\hline
		ca6655a, f84a4db & Сообщение не раскрывает суть изменения, добавьте описание & Сообщение не несет смысла, добавьте подробное описание изменений \\
		\hline
		250fab4 & Несоответствие стилю сообщения коммита, улучшите форматирование & Сообщение очень короткое, пожалуйста, опишите изменения подробнее.\\
		\hline
		6ad7d36, 838f481 & Нарушение атомарности изменений, должны быть разделены & \makecell[tl]{Высокий риск: рекомендуется\\ провести детальный код-ревью.\\Основные факторы риска:\\ message\_length, files\_changed,\\ avg\_file\_history} \\
		\hline
		4e5660e, 0b7863b & Сообщение коммита без контекста, используйте стандарты & Сообщение не информативно, добавьте больше деталей о произведённых изменениях \\
		\hline
		196baa7, 7b707df & Отсутствие тестов для изменений & Добавьте тесты, чтобы подтвердить корректность изменений \\
		\hline
		05caf51, 7b707df & Несоответствие авторства коммита (author и committer разные) & Проверьте соответствие авторства и коммита, исправьте несоответствия \\
		\hline
		7404e85, 838f481 & Наличие бинарных файлов в коммите, что недопустимо для контроля версий & Удалите бинарные файлы из коммита. \\
		\hline
		250fab4, 4e5660e & Проблемы с форматированием сообщения коммита & Сообщение коммита слишком короткое, добавьте более подробное описание изменений \\
		\hline
		7b707df, 0b7863b & Несоответствие форматирования коммита, добавьте контекст & Сообщение слишком короткое, оно не даёт понимания изменений \\
		\hline
	\end{tabularx}
\end{table}

Из сравнения рекомендаций экспертов и модели можно сделать следующие выводы:

Модель предоставляет рекомендации, которые в большинстве случаев соответствуют экспертным меткам. Основные рекомендации, такие как использование fast-forward или squash для merge-коммитов, разделение неатомарных изменений и улучшение сообщений, совпадают.
Модель также выдает полезные предложения по улучшению качества сообщений и увеличению точности изменений, что подтверждает её эффективность в анализе репозиториев.
Несмотря на высокую степень совпадения, есть области, где модель могла бы быть улучшена, например, в более точном учете контекста изменений, что позволило бы избежать некоторых ложных срабатываний.

Система, в целом, эффективно предоставляет рекомендации, которые могут улучшить качество кода в проекте, обеспечив более строгие практики ведения репозиториев.

\section{Выводы}
В результате экспериментального тестирования подтверждена корректность реализации разработанной системы: интерфейс и функциональные блоки работают согласно заданию, а обученный классификатор рисковых коммитов показывает адекватные метрики на реальных данных. Успешно протестирована модель генерации рекомендаций CAPA: выявленные рекомендации соответствуют ожидаемым паттернам исправлений. Модульные тесты покрыли основные пути выполнения, включая граничные сценарии, что свидетельствует о надёжности кода. Нагрузочные тесты показали, что система масштабируема — даже при анализе тысячи коммитов реакция остаётся предсказуемой, хотя оптимизации статического анализа целесообразны для ускорения. Сильными сторонами подхода являются комплексность анализа (объединение статики кода, анализа коммитов и визуализации) и высокая адаптивность модели (универсальный класс классификатора позволяет легко тестировать новые алгоритмы). Подробные визуализации дают мощный инструмент мониторинга проекта. Возможные улучшения: расширение набора признаков за счёт динамических метрик (например, метрики сборки или покрытия тестами), а также использование реальных размеченных данных для обучения классификатора вместо эвристической псевдоразметки. При дальнейшем развитии можно добавить автоматические рекомендации по приоритетам исправлений на основе выявленных рисков. В целом, проведённое экспериментальное тестирование подтвердило, что разработанный подход позволяет эффективно выявлять и анализировать рисковые изменения в репозиториях, обеспечивая корректность работы системы и демонстрируя перспективу для дальнейшего улучшения инструментов разработки.