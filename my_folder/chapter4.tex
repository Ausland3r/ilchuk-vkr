\chapter{Апробация результатов исследования} \label{ch4}

\section{Цель апробации} \label{ch4:sec1}
Апробация разработанного метода включает тестирование модели на реальных данных, оценку её эффективности и сравнение с альтернативными подходами. Основной целью является определение точности, устойчивости и применимости модели в различных сценариях. 

\section{Методика тестирования} \label{ch4:sec2}
Для проверки работоспособности модели были проведены следующие этапы тестирования:
\begin{itemize}
	\item Оценка качества классификации коммитов с использованием метрик (точность);
	\item Анализ временной эффективности работы алгоритма на выборке разного объема;
	\item Сравнение результатов работы модели с альтернативными методами (например, случайный лес);
\end{itemize}


\section{Практическое применение и апробация} \label{ch4:sec3}
Результаты апробации были проверены на нескольких репозиториях с различными паттернами разработки. Для этого были выбраны проекты с:
\begin{itemize}
	\item Высокой частотой коммитов (активно разрабатываемые проекты);
	\item Длинными промежутками между коммитами (поддерживаемые проекты);
	\item Большим количеством изменений в кодовой базе.
\end{itemize}

Анализ показал, что метод корректно адаптируется к различным сценариям, предоставляя полезные рекомендации по улучшению процессов разработки.

\section{Выводы} \label{ch4:conclusion}
Результаты апробации показали, что предложенный метод анализа коммитов позволяет эффективно выявлять потенциальные проблемы в процессе разработки, формируя полезные рекомендации CAPA. Сравнительный анализ подтвердил конкурентоспособность модели относительно существующих методов, а тестирование на реальных данных продемонстрировало её практическую применимость.
