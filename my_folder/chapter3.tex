\chapter{Разработка программного обеспечения} \label{ch3}

\section{Анализ данных из репозитория} \label{ch3:sec1}

Для анализа данных использовался репозиторий на GitHub. Данные включали историю коммитов, информацию об авторах, количество добавленных и удалённых строк кода, изменения в файлах и временные интервалы между коммитами. На основе этих данных разработана система автоматического извлечения информации, её анализа и формирования рекомендаций по корректирующим и предупреждающим действиям (CAPA).

\begin{lstlisting}[language=Python]
# Чтение конфигурации из файла
config = {}
with open('config.txt', 'r') as file:
for line in file:
name, value = line.strip().split('=')
config[name] = value

access_token = config['access_token']
repos = config['repos'].split(',')

# Аутентификация с использованием токена доступа
g = Github(access_token)

# Создание файла для записи данных
with open('repository_data.csv', mode='w', newline='', encoding='utf-8') as file:
writer = csv.writer(file)
writer.writerow(['Repo', 'Commit SHA', 'Author', 'Date', 'Message', 'Additions', 'Deletions', 'Total Changes', 'File Changes', 'Time Since Last Commit', 'Has CAPA'])

for repo_name in repos:
repo = g.get_repo(repo_name)
commits = repo.get_commits()

previous_commit_date = None
for commit in commits:
sha = commit.sha
author = commit.commit.author.name
date = commit.commit.author.date
message = commit.commit.message
stats = commit.stats
additions = stats.additions
deletions = stats.deletions
total_changes = stats.total
files_changed = commit.files

if previous_commit_date:
time_since_last_commit = abs((date - previous_commit_date).days)
else:
time_since_last_commit = 0

previous_commit_date = date
has_capa = 0  # Изначально все коммиты не CAPA
writer.writerow([repo_name, sha, author, date, message, additions, deletions, total_changes, len(files_changed), time_since_last_commit, has_capa])

print("Данные успешно сохранены в repository_data.csv")
\end{lstlisting}


\section{Методы обработки данных и обучение моделей} \label{ch3:sec2}

Система реализована на языке Python и включает следующие этапы:
\begin{itemize}
	\item Извлечение данных о коммитах из репозитория GitHub.
	\item Запись данных в CSV-файл для последующего анализа.
	\item Обработка данных и определение пороговых значений с использованием алгоритма кластеризации KMeans.
	\item Обучение модели для определения CAPA с использованием алгоритмов машинного обучения (случайный лес и наивный байесовский классификатор).
	\item Генерация рекомендаций CAPA на основе анализа данных.
	\item Запись рекомендаций в файл и автоматическое создание pull request на GitHub.
	\item Визуализация данных с использованием интерактивного дашборда на основе библиотеки Dash.
\end{itemize}

Для выявления аномалий использовался метод кластеризации KMeans, который позволил установить пороговые значения на основе характеристик коммитов (количество добавленных и удалённых строк, изменения в файлах, временные интервалы между коммитами и другие параметры). 

Далее проводилось обучение моделей машинного обучения, включая случайный лес и наивный байесовский классификатор. Экспериментальные результаты показали, что случайный лес обеспечил более высокую точность классификации (1.0), по сравнению с наивным байесовским классификатором (0.8), что связано с возможными сложными зависимостями между признаками и наличием выбросов в данных.

\begin{lstlisting}[language=Python]
# Загрузка данных из файла
data = pd.read_csv('repository_data.csv', encoding='utf-8')

# Использование KMeans для определения пороговых значений
features = ['Additions', 'Deletions', 'Total Changes', 'File Changes', 'Time Since Last Commit']
kmeans = KMeans(n_clusters=2, random_state=42)
data['Cluster'] = kmeans.fit_predict(data[features])

# Вычисление пороговых значений на основе кластеров
cluster_centers = kmeans.cluster_centers_
ADDITION_THRESHOLD = cluster_centers[:, 0].mean()
DELETION_THRESHOLD = cluster_centers[:, 1].mean()
TOTAL_CHANGE_THRESHOLD = cluster_centers[:, 2].mean()
FILE_CHANGE_THRESHOLD = cluster_centers[:, 3].mean()
TIME_SINCE_LAST_COMMIT_THRESHOLD = cluster_centers[:, 4].mean()
\end{lstlisting}

\section{Интеграция модели глубокого леса} \label{ch3:sec3}

Для дальнейшего улучшения точности была внедрена модель глубокого леса (Deep Forest). Так как библиотека deep-forest несовместима с Python 3.12, была создана виртуальная среда на Python 3.9 для выполнения модели. Основная программа вызывала исполняемый файл в этой среде, передавала данные из репозитория и получала обработанные результаты.

Глубокий лес был обучен на тех же данных, что и предыдущие модели, но позволил более точно выявлять сложные зависимости между параметрами коммитов. Итоговые предсказания загружались обратно в основной процесс для последующего анализа.
\begin{lstlisting}[language=Python]
# Сохранение важностей признаков в CSV
importance_df = pd.DataFrame({'Feature': features, 'Importance': rf_model.feature_importances_})
importance_df.to_csv('feature_importances.csv', index=False)

input_csv = 'repository_data.csv'
output_csv = 'repository_data_with_predictions.csv'
python_venv = os.path.join(os.getcwd(), 'myenv', 'Scripts', 'python.exe')
if not os.path.exists(python_venv):
print(f"Virtual environment not found at {python_venv}")
sys.exit(1)

result = subprocess.run([python_venv, 'deep_forest_task.py', input_csv, output_csv], capture_output=True, text=True)
print(result.stdout)
print(result.stderr, file=sys.stderr)

if result.returncode == 0:
data_with_predictions = pd.read_csv(output_csv)
print("Результаты предсказаний загружены успешно.")
else:
print("Ошибка при выполнении подзадачи.")
\end{lstlisting}


\section{Генерация рекомендаций CAPA} \label{ch3:sec4}

На основе адаптивных пороговых значений для каждого коммита формировались рекомендации CAPA. Если одна из метрик превышала вычисленные пороговые значения, создавалась рекомендация, которая добавлялась в итоговый список. 

Пример рекомендаций:
\begin{itemize}
	\item Если количество добавленных строк превышает порог, рекомендуется провести ревизию кода.
	\item Если количество удалённых строк значительно увеличилось, это может указывать на устранение проблем или рефакторинг кода.
	\item Если временной интервал между коммитами велик, это может свидетельствовать о задержке в разработке, требующей пересмотра приоритетов.
\end{itemize}

\begin{lstlisting}[language=Python]
# Функция генерации рекомендаций CAPA
def generate_capa_recommendations(data, ADDITION_THRESHOLD, DELETION_THRESHOLD, TOTAL_CHANGE_THRESHOLD, FILE_CHANGE_THRESHOLD, TIME_SINCE_LAST_COMMIT_THRESHOLD):
recommendations = []
for index, commit_data in data.iterrows():
suggestion = []
if commit_data['Additions'] > ADDITION_THRESHOLD:
suggestion.append(' | Review large additions |')
if commit_data['Deletions'] > DELETION_THRESHOLD:
suggestion.append(' | Review large deletions (Many deletions may indicate a fix) |')
if commit_data['Total Changes'] > TOTAL_CHANGE_THRESHOLD:
suggestion.append(' | Review total changes for potential issues (Too many changes since last commit) |')
if commit_data['File Changes'] > FILE_CHANGE_THRESHOLD:
suggestion.append(' | Review changes in multiple files (Many files updated, expected compatibility issues) |')
if abs(commit_data['Time Since Last Commit']) > abs(TIME_SINCE_LAST_COMMIT_THRESHOLD):
suggestion.append('| Review large time gap between commits (Large time gap between commits may indicate a slow development process) |')
if len(commit_data['Message']) < 15 and commit_data['Total Changes'] > TOTAL_CHANGE_THRESHOLD:
suggestion.append('| Commit message is too short, consider providing more details |')
if not suggestion:
suggestion.append('No specific recommendations')

recommendations.append({
	'Repo': commit_data['Repo'],
	'Commit SHA': commit_data['Commit SHA'],
	'Message': commit_data['Message'],
	'Suggestion': '; '.join(suggestion)
})

return recommendations

# Выявление аномалий и генерация рекомендаций
recommendations = generate_capa_recommendations(data_with_predictions, ADDITION_THRESHOLD, DELETION_THRESHOLD, TOTAL_CHANGE_THRESHOLD, FILE_CHANGE_THRESHOLD, TIME_SINCE_LAST_COMMIT_THRESHOLD)
recommendations_df = pd.DataFrame(recommendations)
\end{lstlisting}

\section{Разработка интерактивного дашборда} \label{ch3:sec5}

Для визуализации результатов анализа был разработан дашборд на основе библиотеки Dash. Дашборд содержит:
\begin{itemize}
	\item Графики активности разработчиков.
	\item Динамику изменений кода.
	\item Выявленные аномалии и рекомендации CAPA.
	\item Историю коммитов с классификацией по типам изменений.
\end{itemize}

Это позволило наглядно представлять анализируемые данные и упрощать процесс принятия решений по улучшению качества кода.

\section{Выводы} \label{ch3:conclusion}

В данной главе рассмотрена реализация системы автоматического анализа коммитов. Использование методов машинного обучения и кластеризации позволило выявлять аномалии в процессе разработки и формировать рекомендации по корректирующим действиям. 

Интеграция модели глубокого леса позволила повысить точность классификации, а интерактивный дашборд упростил процесс мониторинга состояния репозитория. Таким образом, предложенная система способствует повышению качества кода и автоматизации процесса контроля за разработкой программного обеспечения.
