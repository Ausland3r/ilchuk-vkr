\chapter*{Введение} % * не проставляет номер
\addcontentsline{toc}{chapter}{Введение} % вносим в содержание

\textbf{Актуальность исследования.} Современные разработки программного обеспечения (ПО) становятся всё более сложными, требуя высоких стандартов качества и точного контроля за процессом разработки. В условиях глобализации и быстрого развития технологий компании стремятся не только создавать качественные продукты, но и эффективно управлять процессом их создания. Однако существующие подходы к анализу и управлению качеством ПО требуют значительных временных и человеческих ресурсов. Методы анализа данных и машинного обучения, которые уже зарекомендовали себя в смежных областях, могут быть использованы для автоматизации контроля качества и выявления проблем на ранних этапах. Одной из ключевых задач в этой области является анализ данных из репозиториев исходного кода, таких как GitHub.

Коммиты в репозиториях содержат важную информацию о внесённых изменениях: количество добавленных и удалённых строк кода, изменённые файлы, временные интервалы между изменениями. Анализ этих данных позволяет выявить потенциальные отклонения от нормального процесса разработки и предложить корректирующие и предупреждающие действия (CAPA). Несмотря на широкий спектр существующих инструментов для анализа данных из репозиториев, большинство из них либо недостаточно автоматизированы, либо не позволяют выявлять комплексные закономерности в данных.

\textbf{Цель исследования:} разработка системы, которая позволит автоматизировать процесс анализа коммитов и извлечения CAPA на основе методов машинного обучения и кластеризации.

\textbf{Задачи исследования:}
\begin{itemize}
	\item Провести обзор существующих методов анализа данных из репозиториев кода.
	\item Изучить применимость методов кластеризации и алгоритмов машинного обучения для анализа коммитов.
	\item Разработать систему для автоматического извлечения данных о коммитах из нескольких репозиториев GitHub.
	\item Реализовать механизм выявления аномалий и классификации коммитов на основе предложенных методов.
	\item Создать интерактивный дашборд для визуализации результатов анализа.
	\item Оценить эффективность предложенного подхода на реальных данных.
\end{itemize}

Подробнее актуальность исследования и обзор методов рассмотрены в разделе \ref{ch1:sec1}.

Таким образом, исследование направлено на решение задачи повышения эффективности управления качеством программного обеспечения за счёт использования современных технологий анализа данных. Предложенная система должна не только автоматизировать процесс анализа данных, но и предоставлять разработчикам полезные рекомендации для улучшения качества кода и предотвращения потенциальных проблем.
