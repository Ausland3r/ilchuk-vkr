\chapter*{Словарь терминов}             % Заголовок
\addcontentsline{toc}{chapter}{Словарь терминов}  % Добавляем его в оглавление

\textbf{CAPA (Corrective and Preventive Actions)} --- корректирующие и предупреждающие действия, направленные на устранение и предотвращение дефектов в процессе разработки программного обеспечения.

\textbf{GitHub} --- веб-сервис для хостинга IT-проектов и их совместной разработки на базе системы управления версиями Git.

\textbf{Коммит (commit)} --- фиксация изменений в репозитории Git, включающая информацию о внесённых правках, авторе и времени изменения.

\textbf{KMeans} --- метод кластеризации данных, основанный на разбиении множества на k групп по схожести признаков.

\textbf{Случайный лес (Random Forest)} --- ансамблевый метод машинного обучения, использующий множество деревьев решений для повышения точности прогнозов.

\textbf{Наивный Байесовский классификатор} --- алгоритм машинного обучения, основанный на теореме Байеса и предположении независимости признаков.

\textbf{Глубокий лес (Deep Forest)} --- метод машинного обучения, использующий каскадную структуру случайных лесов для улучшения классификации.

\textbf{API (Application Programming Interface)} --- интерфейс программирования приложений, позволяющий взаимодействовать с внешними сервисами и библиотеками.

\textbf{Pull Request (PR)} --- запрос на внесение изменений в репозиторий GitHub, который проходит процесс ревью перед слиянием в основную ветку.

\textbf{Dash} --- фреймворк на Python для создания интерактивных дашбордов и веб-приложений для визуализации данных.
