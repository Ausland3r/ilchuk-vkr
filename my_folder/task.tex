%%%% Начало оформления заголовка - оставить без изменений !!! %%%%
\thispagestyle{empty}%
\setcounter{tskPageFirst}{\value{page}} %сохранили номер первой страницы Задания
\ifnumequal{\value{tskPrint}}{1}{% если двухсторонняя печать Задания, то...
	\newgeometry{twoside,top=2cm,bottom=2cm,left=3cm,right=1cm,headsep=0cm,footskip=0cm}
	\savegeometry{MyTask} %save settings
	\makeatletter % задаём оформление второй страницы ВКР как нечетной, а третьей - как чётной
	\checkoddpage % проверка четности из memoir-класса
	\ifoddpage
	\else
		\let\tmp\oddsidemargin
		\let\oddsidemargin\evensidemargin
		\let\evensidemargin\tmp
		\reversemarginpar
	\fi
	\makeatother
}{} % 
\pagestyle{empty} % удаляем номер страницы на втором/третьем листе
\makeatletter
\newrefcontext[labelprefix={3.}] % задаём префикс для списка литературы
\makeatother
\setlength{\parindent}{0pt}
{\centering\bfseries%
%	\small	% настройки - начало 

\normalfont%
{\centering%
	\Ministry\\
	\SPbPU\\
	{%\bfseries %2020 - указание на изменения, которые могут быть введены в 2020 году
		\institute}
	\par}%\intervalS% завершает input

\noindent
\begin{minipage}{\linewidth}
\vspace{\mfloatsep} % интервал 	
\begin{tabularx}{\linewidth}{Xl}
	&УТВЕРЖДАЮ      \\
	&\HeadTitle     \\			
	&\underline{\hspace*{0.1\textheight}} \Head     \\
	&<<\underline{\hspace*{0.05\textheight}}>> \underline{\hspace*{0.1\textheight}} \thesisYear г.  \\  
\end{tabularx}
\vspace{\mfloatsep} % интервал 	
\end{minipage}

\intervalS{\centering\bfseries%

ЗАДАНИЕ\\
на выполнение %с 2020 года 
%по выполнению % до 2020 года
выпускной квалификационной работы}


\intervalS\normalfont%

студенту \uline{\AuthorFullDat{} гр.~\group}


\par}\intervalS%
%%%%
%%%% Конец оформления заголовка  %%%%



\begin{enumerate}[1.]
\item Тема работы: {\expandafter \thesisTitle.}
%\item Тема работы (на английском языке): \uline{\thesisTitleEn.} % вероятно после 2021 года
\item Срок сдачи студентом законченной работы: \uline{\thesisDeadline.} 
\item Исходные данные по работе: \begin{enumerate}[label=\theenumi\arabic*.]
	\item An experience in automatically extracting CAPAs from code repositories: [Электронный ресурс]. URL: \url{https://arxiv.org/pdf/2212.09910} (дата обращения: 12.12.2024).
	\item A meta-analytical comparison of Naive Bayes and Random Forest for software defect prediction: [Электронный ресурс]. URL: \url{https://www.researchgate.net/publication/350459831_A_metaanalytical_comparison_of_Naive_Bayes_and_Random_Forest_for_software_defect_prediction} (дата обращения: 12.12.2024).
	\item Examining the Success of an Open Source Software Project Analysing Its Repository: [Электронный ресурс]. URL: \url{https://doi.org/10.5281/zenodo.10046579} (дата обращения: 12.12.2024).
	\item Github API documentation: [Электронный ресурс]. URL: \url{https://docs.github.com/en/rest?apiVersion=2022-11-28} (дата обращения: 12.12.2024).
	\item PyGithub documentation: [Электронный ресурс]. URL: \url{https://pygithub.readthedocs.io/en/stable/} (дата обращения: 12.12.2024).
	\item The k-means Algorithm: A Comprehensive Survey and Performance Evaluation: [Электронный ресурс]. URL: \url{https://www.mdpi.com/2079-9292/9/8/1295} (дата обращения: 12.12.2024).
\end{enumerate}
%}%
\printbibliographyTask % печать списка источников % КОММЕНТИРУЕМ ЕСЛИ НЕ ИСПОЛЬЗУЕТСЯ
% В СЛУЧАЕ, ЕСЛИ НЕ ИСПОЛЬЗУЕТСЯ МОЖНО ТАКЖЕ ЗАЙТИ В setup.tex и закомментировать \vspace{-0.28\curtextsize}
\item Содержание работы (перечень подлежащих разработке вопросов):
\begin{enumerate}[label=\theenumi\arabic*.]
\item Исследование методов и средств формирования CAPAs на основе изменений репозитория кода.
\item Проектирование системы формирования CAPAs на основе изменений репозитория кода.
\item Реализация системы формирования CAPAs на основе изменений репозитория кода.
\item Тестирование и апробация системы формирования CAPAs на основе изменений репозитория кода.
\end{enumerate}
\item Перечень графического материала (с указанием обязательных чертежей): 
\begin{enumerate}[label=\theenumi\arabic*.]
\item Сравнительная таблица средств CAPA.
\item Диаграмма вариантов использования.
\item Функциональная модель IDEF0.
\item Архитектура разработанной программы.
\item Диаграмма классов.
\item Графики и таблицы с результатами модульного и нагрузочного тестирования, включая оценку качества тестов, результатами экспериментов, сравнения разных алгоритмов машинного обучения.
\end{enumerate}
\item Перечень используемых информационных технологий, в том числе программное обеспечение, облачные сервисы, базы данных и прочие сквозные цифровые технологии (при наличии):
\begin{enumerate}[label=\theenumi\arabic*.]
\item Python, Git, Github.
\item Pandas, NumPy, Scikit-learn.
\item Plotly, Dash.
\end{enumerate}
\item Консультанты по работе:
\begin{enumerate}[label=\theenumi\arabic*.] 
\item {\emakefirstuc{\ConsultantNormDegree}, \ConsultantNorm{} (нормоконтроль).} %	Обязателен для всех студентов
\end{enumerate}
\item Дата выдачи задания: {\thesisStartDate.}
\end{enumerate}

\intervalS%можно удалить пробел

Руководитель ВКР {\hspace*{0.1\textheight} \Supervisor}

%Консультант по нормоконтролю \uline{\hspace*{0.1\textheight} \ConsultantNorm}%ПОКА НЕ ТРЕБУЕТСЯ, Т.К. ОН У ВСЕХ ПО УМОЛЧАНИЮ

Задание принял к исполнению {\thesisStartDate}

\intervalS%можно удалить пробел

Студент {\hspace*{0.1\textheight}  \Author}



%\setcounter{tskPageLast}{\value{page}} %сохранили номер последней страницы Задания
\setcounter{tskPages}{\value{tskPageLast}-\value{tskPageFirst}}
\newrefsection % начинаем новую секцию библиографии
\newrefcontext % удаляем префикс к пунктам списка литературы
\restoregeometry % восстанавливаем настройки страницы
\pagestyle{plain} % удаляем номер страницы на первой/второй странице Задания
\setlength{\parindent}{2.5em} % восстанавливаем абзацный отступ
%% Обязательно закомментировать, если получается один лист в задании:
\ifnumequal{\value{tskPages}}{0}{% Если 1 страница в Задании, то ничего не делать.
}{% Иначе 
% до 2020 года требовалось печатать задание на 1 листе с двух сторон и не подсчитывать вторую страницу
%\setcounter{page}{\value{page}-\value{tskPages}} 	% вычесть значение tskPages при печати более 1 страницы страниц
}%
\AtNextBibliography{\setcounter{citenum}{0}}%обнуляем счетчик библиографии	% настройки - конец
