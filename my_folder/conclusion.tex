\chapter*{Заключение} \label{ch-conclusion}
\addcontentsline{toc}{chapter}{Заключение} % в оглавление

В ходе данной работы была разработана система автоматического анализа коммитов, направленная на выявление аномалий и формирование корректирующих и предупреждающих действий (CAPA). Использование методов машинного обучения и кластеризации позволило создать инструмент, способный анализировать историю изменений в коде и предлагать рекомендации для повышения качества программного обеспечения.

Основные результаты работы можно сформулировать следующим образом:
\begin{itemize}
	\item Проведен обзор существующих методов анализа данных из репозиториев исходного кода и выявлены их ограничения.
	\item Разработан алгоритм автоматического извлечения данных о коммитах с последующей их обработкой и анализом.
	\item Предложен метод кластеризации коммитов с использованием алгоритма KMeans для определения пороговых значений изменений в коде.
	\item Обучены и протестированы модели машинного обучения (случайный лес, наивный байесовский классификатор и глубокий лес), показавшие высокую точность в задаче предсказания аномалий.
	\item Разработан механизм автоматического создания pull request с рекомендациями CAPA, который интегрируется в процесс разработки.
	\item Создан интерактивный дашборд для визуализации результатов анализа, что позволяет разработчикам легко отслеживать состояние репозитория и принимать решения на основе данных.
\end{itemize}

Практическая значимость предложенной системы заключается в том, что она позволяет автоматизировать контроль за качеством кода, минимизировать ошибки, возникающие в процессе разработки, и повысить прозрачность изменений в репозитории. Используемый подход может быть адаптирован для различных проектов и масштабируем для работы с крупными кодовыми базами.

В дальнейшем возможны следующие направления развития системы:
\begin{itemize}
	\item Доработка алгоритмов выявления аномалий с учетом более сложных паттернов изменений в коде.
	\item Расширение набора метрик для анализа коммитов.
	\item Интеграция с другими инструментами контроля качества кода и CI/CD системами.
	\item Применение нейросетевых моделей для улучшения предсказательной способности системы.
\end{itemize}

Таким образом, проведенное исследование подтвердило эффективность предложенного подхода к анализу коммитов. Разработанная система способствует улучшению управления процессом разработки программного обеспечения, сокращает время на выявление потенциальных проблем и повышает качество выпускаемого кода.
