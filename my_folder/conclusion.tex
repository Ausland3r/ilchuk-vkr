\chapter*{Заключение} \label{ch-conclusion}
\addcontentsline{toc}{chapter}{Заключение} % в оглавление

В ходе данной работы была разработана система автоматического анализа коммитов, направленная на выявление аномалий и формирование корректирующих и предупреждающих действий. Использование методов машинного обучения и кластеризации позволило создать инструмент, способный анализировать историю изменений в коде и предлагать рекомендации для повышения качества программного обеспечения.

Основные результаты работы можно сформулировать следующим образом:
\begin{itemize}
	\item Проведен обзор существующих методов анализа данных из репозиториев исходного кода и выявлены их ограничения.
	\item Разработан алгоритм автоматического извлечения данных о коммитах с последующей их обработкой и анализом.
	\item Предложен метод кластеризации коммитов с использованием алгоритма KMeans для определения пороговых значений изменений в коде.
	\item Обучены и протестированы модели машинного обучения (случайный лес, наивный байесовский классификатор и глубокий лес), показавшие высокую точность в задаче предсказания аномалий.
	\item Разработан механизм автоматического создания pull request с рекомендациями CAPA, который интегрируется в процесс разработки.
	\item Создан интерактивный дашборд для визуализации результатов анализа, что позволяет разработчикам легко отслеживать состояние репозитория и принимать решения на основе данных.
\end{itemize}

Практическая значимость предложенной системы заключается в том, что она позволяет автоматизировать контроль за качеством кода, минимизировать ошибки, возникающие в процессе разработки, и повысить прозрачность изменений в репозитории. Используемый подход может быть адаптирован для различных проектов и масштабируем для работы с крупными кодовыми базами.

В дальнейшем возможны следующие направления развития системы:
\begin{itemize}
	\item Доработка алгоритмов выявления аномалий с учетом более сложных паттернов изменений в коде.
	\item Расширение набора метрик для анализа коммитов.
	\item Интеграция с другими инструментами контроля качества кода и CI/CD системами.
	\item Применение нейросетевых моделей для улучшения предсказательной способности системы.
\end{itemize}

Таким образом, проведенное исследование подтвердило эффективность предложенного подхода к анализу коммитов. Разработанная система способствует улучшению управления процессом разработки программного обеспечения, сокращает время на выявление потенциальных проблем и повышает качество выпускаемого кода.


\chapter*{Приложение} \label{ch5}

\begin{lstlisting}[language=Python, caption={{ \texttt{repository\_analysis.py}}}]
	# repository_analysis.py
	
	import os
	import requests
	import re
	import json
	import subprocess
	from datetime import datetime
	import pandas as pd
	from git import Repo, GitCommandError
	from typing import List, Dict, Optional
	
	from xgboost import XGBClassifier
	
	from ml_model import CommitRiskModel
	from recommendations import generate_recommendations
	
	LANGUAGE_ANALYZERS = {
		'.py': 'python',
		'.js': 'javascript',
		'.ts': 'javascript',
		'.java': 'java',
	}
	
	class GitHubRepoAnalyzer:
	def __init__(
	self,
	repo_owner: str,
	repo_name: str,
	token: str,
	clone_dir: str = "/tmp",
	):
	self.repo_owner = repo_owner
	self.repo_name = repo_name
	self.token = token
	self.api_url = f"https://api.github.com/repos/{repo_owner}/{repo_name}"
	self.headers = {"Authorization": f"token {token}"}
	
	self.local_path = os.path.join(clone_dir, repo_name)
	if not os.path.isdir(self.local_path):
	clone_url = f"https://github.com/{repo_owner}/{repo_name}.git"
	print(f"[INIT] Cloning repository {clone_url} into {self.local_path}")
	Repo.clone_from(clone_url, self.local_path)
	print(f"[INIT] Clone complete.")
	else:
	print(f"[INIT] Repository already cloned at {self.local_path}.")
	self.repo = Repo(self.local_path)
	print(f"[INIT] Repo object ready at {self.local_path}.")
	
	self.complexity_re = re.compile(r"\b(if|for|while|switch|case)\b")
	
	def get_commits(self) -> List[Dict]:
	print("[COMMITS] Fetching commits via GitHub API")
	commits, page, per_page = [], 1, 100
	while True:
	print(f"[COMMITS] Requesting page {page}")
	resp = requests.get(
	f"{self.api_url}/commits",
	headers=self.headers,
	params={"page": page, "per_page": per_page},
	)
	data = resp.json()
	if resp.status_code == 401:
	raise RuntimeError("Bad credentials: check your GITHUB_TOKEN")
	if not isinstance(data, list):
	print(f"[COMMITS] Unexpected response: {data}")
	break
	commits.extend(data)
	print(f"[COMMITS] Retrieved {len(data)} commits in page {page}.")
	if len(data) < per_page:
	print(f"[COMMITS] Less than {per_page} commits on page {page}, finishing.")
	break
	page += 1
	print(f"[COMMITS] Total commits fetched: {len(commits)}")
	return commits
	
	def get_commit_details(self, sha: str) -> Dict:
	print(f"[DETAILS] Fetching details for commit {sha}")
	resp = requests.get(f"{self.api_url}/commits/{sha}", headers=self.headers)
	return resp.json()
	
	def detect_language(self, filename: str) -> str:
	_, ext = os.path.splitext(filename.lower())
	return LANGUAGE_ANALYZERS.get(ext, "")
	
	def analyze_python_file(self, full_path: str) -> Dict[str,int]:
	pyl_w = pyl_e = bandit = 0
	try:
	r = subprocess.run(
	["pylint", full_path, "--output-format=json"],
	stdout=subprocess.PIPE, stderr=subprocess.DEVNULL, text=True
	)
	msgs = json.loads(r.stdout or "[]")
	for m in msgs:
	if m.get("type") == "error":
	pyl_e += 1
	else:
	pyl_w += 1
	except Exception:
	print(f"[ANALYZE][PY] Pylint failed on {full_path}")
	try:
	r = subprocess.run(
	["bandit", "-f", "json", "-r", full_path],
	stdout=subprocess.PIPE, stderr=subprocess.DEVNULL, text=True
	)
	js = json.loads(r.stdout or "{}")
	bandit = len(js.get("results", []))
	except Exception:
	print(f"[ANALYZE][PY] Bandit failed on {full_path}")
	return {"pylint_warnings": pyl_w, "pylint_errors": pyl_e, "bandit_issues": bandit}
	
	def analyze_javascript_file(self, full_path: str) -> Dict[str,int]:
	w = e = 0
	try:
	r = subprocess.run(
	["eslint", full_path, "-f", "json"],
	stdout=subprocess.PIPE, stderr=subprocess.DEVNULL, text=True
	)
	arr = json.loads(r.stdout or "[]")
	for file_res in arr:
	for msg in file_res.get("messages", []):
	if msg.get("severity") == 2:
	e += 1
	else:
	w += 1
	except Exception:
	print(f"[ANALYZE][JS] ESLint failed on {full_path}")
	return {"eslint_warnings": w, "eslint_errors": e}
	
	def analyze_java_file(self, full_path: str) -> Dict[str,int]:
	count = 0
	try:
	r = subprocess.run(
	["checkstyle", "-f", "plain", full_path],
	stdout=subprocess.PIPE, stderr=subprocess.DEVNULL, text=True
	)
	for ln in r.stdout.splitlines():
	if "ERROR" in ln or "WARNING" in ln:
	count += 1
	except Exception:
	print(f"[ANALYZE][JAVA] Checkstyle failed on {full_path}")
	return {"checkstyle_issues": count}
	
	def compute_repo_stats(self, commits: List[Dict]) -> Dict:
	import pandas as pd
	df = pd.DataFrame(commits)
	stats = {}
	for f in ['lines_added', 'lines_deleted', 'files_changed',
	'avg_file_history', 'message_length', 'complexity_score']:
	if f in df:
	stats[f] = {
		'mean': df[f].mean(),
		'std': df[f].std(),
		'quantile_90': df[f].quantile(0.90),
		'quantile_95': df[f].quantile(0.95),
	}
	stats['author_stats'] = {a: {'median_lines_added': grp.median()}
		for a, grp in df.groupby('author_name')['lines_added']}
	if 'minutes_since_previous_commit' in df:
	stats['commit_interval'] = {'median': df['minutes_since_previous_commit'].median()}
	return stats
	
	def analyze_commits(self) -> List[Dict]:
	print("[ANALYZE]Starting commit-by-commit analysis")
	commits_data, file_count = [], {}
	
	all_commits = self.get_commits()
	all_commits.reverse()
	prev_dt = None
	
	for idx, c in enumerate(all_commits, 1):
	sha = c["sha"]
	print(f"[ANALYZE] ({idx}/{len(all_commits)}) Processing commit {sha}")
	det = self.get_commit_details(sha)
	
	try:
	print(f"[GIT] Checking out {sha}")
	self.repo.git.checkout(sha)
	except GitCommandError:
	print(f"[GIT] Cannot checkout {sha}, skipping FS analysis")
	
	msg = det["commit"]["message"]
	author = det["commit"]["author"]
	name = author.get("name", "Unknown")
	dt = datetime.strptime(author["date"], "%Y-%m-%dT%H:%M:%SZ")
	
	files = det.get("files", [])
	print(f"[ANALYZE]  {len(files)} files changed")
	
	added = sum(f.get("additions", 0) for f in files)
	deleted = sum(f.get("deletions", 0) for f in files)
	hist = sum(file_count.get(f["filename"], 0) for f in files)
	avg_hist = hist / len(files) if files else 0
	
	comp = 0
	for f in files:
	for ln in f.get("patch", "").splitlines():
	if ln.startswith("+") and not ln.startswith("+++") and self.complexity_re.search(ln):
	comp += 1
	
	delta = (dt - prev_dt).total_seconds() / 60 if prev_dt else None
	
	metrics = {k: 0 for k in (
		"pylint_warnings","pylint_errors","bandit_issues",
		"eslint_warnings","eslint_errors","checkstyle_issues"
		)}
	for f in files:
	lang = self.detect_language(f["filename"])
	full = os.path.join(self.local_path, f["filename"])
	if lang:
	print(f"[ANALYZE]Running {lang} analysis on {f['filename']}")
	if lang == "python":
	out = self.analyze_python_file(full)
	elif lang == "javascript":
	out = self.analyze_javascript_file(full)
	elif lang == "java":
	out = self.analyze_java_file(full)
	else:
	out = {}
	for k,v in out.items():
	metrics[k] += v
	
	data = {
		"commit": sha,
		"author_name": name,
		"author_datetime": dt,
		"minutes_since_previous_commit": delta,
		"message": msg,
		"message_length": len(msg),
		"lines_added": added,
		"lines_deleted": deleted,
		"files_changed": len(files),
		"avg_file_history": avg_hist,
		"complexity_score": comp,
		"file_list": [f["filename"] for f in files],
		**metrics
	}
	commits_data.append(data)
	
	for f in files:
	file_count[f["filename"]] = file_count.get(f["filename"], 0) + 1
	prev_dt = dt
	
	print(f"[ANALYZE] Completed analysis of {len(commits_data)} commits.")
	return commits_data
	
	def create_capa_file(self, commits: List[Dict]) -> str:
	path = os.path.join(self.local_path, "CapaRecommendations.md")
	with open(path, "w", encoding="utf-8") as f:
	f.write("CAPA Recommendations\n\n")
	for c in commits:
	if c.get("has_capa"):
	for rec in c["capa_recommendations"]:
	f.write(f"- {rec}\n")
	f.write("\n")
	return path
	
	def push_and_create_pr(self, branch_name: str, file_path: str) -> None:
	print(f"[PR] Fetching origin")
	self.repo.git.fetch('origin')
	
	base_branch = 'main'
	
	if branch_name in self.repo.branches:
	print(f"[PR] Branch {branch_name} already exists locally, checking out.")
	self.repo.git.checkout(branch_name)
	else:
	print(f"[PR] Creating branch {branch_name} from origin/{base_branch}")
	self.repo.git.checkout('-b', branch_name, f'origin/{base_branch}')
	
	rel_path = os.path.relpath(file_path, self.local_path)
	print(f"[PR] Adding file {rel_path}")
	self.repo.index.add([rel_path])
	print(f"[PR] Committing changes")
	self.repo.index.commit("Add CAPA recommendations")
	
	print(f"[PR] Pushing branch {branch_name}")
	origin = self.repo.remote(name='origin')
	origin.push(branch_name)
	
	pr_data = {
		"title": "Add automated CAPA recommendations",
		"head": f"{self.repo_owner}:{branch_name}",
		"base": base_branch,
		"body": "This PR adds automatically generated corrective/preventive actions."
	}
	print(f"[PR] Opening PR via GitHub API")
	response = requests.post(
	f"{self.api_url}/pulls",
	headers=self.headers,
	json=pr_data
	)
	if response.status_code in (200, 201):
	pr_url = response.json().get("html_url")
	print(f"[PR] Pull request created: {pr_url}")
	else:
	print(f"[PR]Failed to create PR: {response.status_code} {response.text}")
	
	def analyze_and_pr(self, commits: Optional[List[Dict]] = None) -> None:
	if commits is None:
	commits = self.analyze_commits()
	
	if not commits:
	print("No commits - пропускаем PR.")
	return
	
	model = CommitRiskModel(classifier=XGBClassifier(eval_metric="logloss"))
	model.fit(commits)
	probs = model.predict_proba(commits)
	
	repo_stats = self.compute_repo_stats(commits)
	
	for commit, p in zip(commits, probs):
	commit["Risk_Proba"] = float(p)
	commit["has_capa"] = True
	commit["capa_recommendations"] = generate_recommendations(
	commit, p, repo_stats, model.feature_importances()
	)
	
	md_path = self.create_capa_file(commits)
	branch = f"capa-{datetime.utcnow():%Y%m%d%H%M}"
	self.push_and_create_pr(branch, md_path)
\end{lstlisting}

\begin{lstlisting}[language=Python, caption={{ \texttt{ml\_model.py}}}]
	# ml_models.py
	from collections import Counter
	from typing import List, Dict, Any, Optional
	import numpy as np
	from deepforest import CascadeForestClassifier
	from sklearn.base import ClassifierMixin
	from sklearn.cluster import KMeans
	from sklearn.inspection import permutation_importance
	from sklearn.metrics import precision_score, recall_score, f1_score, roc_auc_score
	from sklearn.model_selection import train_test_split
	
	
	class CommitRiskModel:
	def __init__(
	self,
	classifier: ClassifierMixin,
	features: Optional[List[str]] = None,
	cluster_model: Optional[KMeans] = None
	):
	self.classifier = classifier
	self.features = features or [
	'lines_added', 'lines_deleted', 'files_changed',
	'avg_file_history', 'message_length',
	'has_bug_keyword', 'complexity_score'
	]
	self.cluster_model = cluster_model or KMeans(n_clusters=2, random_state=0, n_init=10)
	self._is_fitted = False
	self._X: Optional[np.ndarray] = None
	self._y: Optional[np.ndarray] = None
	
	def _extract_X(self, commits: List[Dict[str, Any]]) -> np.ndarray:
	return np.array([[commit.get(f, 0) for f in self.features] for commit in commits])
	
	def _generate_pseudo_labels(self, X: np.ndarray) -> np.ndarray:
	labels = self.cluster_model.fit_predict(X)
	centers = self.cluster_model.cluster_centers_
	
	dist0 = np.linalg.norm(X - centers[0], axis=1)
	dist1 = np.linalg.norm(X - centers[1], axis=1)
	
	prob_cluster1 = dist0 / (dist0 + dist1 + 1e-8)
	
	if centers[0, 0] > centers[1, 0]:
	prob_risky = 1 - prob_cluster1
	else:
	prob_risky = prob_cluster1
	
	threshold = 0.3
	labels_soft = (prob_risky >= threshold).astype(int)
	
	return labels_soft
	
	def fit(self, commits: List[Dict[str, Any]]):
	X = self._extract_X(commits)
	y = self._generate_pseudo_labels(X)
	self.classifier.fit(X, y)
	self._X, self._y = X, y
	self._is_fitted = True
	return self
	
	def predict(self, commits: List[Dict[str, Any]]) -> np.ndarray:
	assert self._is_fitted, "Модель не обучена"
	X = self._extract_X(commits)
	return self.classifier.predict(X)
	
	def predict_proba(self, commits: List[Dict[str, Any]]) -> np.ndarray:
	assert self._is_fitted, "Модель не обучена"
	X = self._extract_X(commits)
	return self.classifier.predict_proba(X)[:, 1]
	
	def feature_importances(self) -> Dict[str, float]:
	if not self._is_fitted or self._X is None or self._y is None:
	raise RuntimeError("Нужно вызвать .fit() перед feature_importances()")
	if hasattr(self.classifier, "feature_importances_"):
	vals = self.classifier.feature_importances_
	else:
	result = permutation_importance(
	self.classifier, self._X, self._y,
	n_repeats=5, random_state=0, n_jobs=-1
	)
	vals = result.importances_mean
	return dict(zip(self.features, vals))
	
	def evaluate_model(self, commits: List[Dict[str, Any]]) -> Dict[str, float]:
	X = self._extract_X(commits)
	y = self._generate_pseudo_labels(X)
	print("[DEBUG] Метки (y) распределение:", Counter(y))
	
	if len(set(y)) < 2:
	print("[WARNING] В данных только один класс, метрики классификации не применимы.")
	clf = self.classifier
	clf.fit(X, y)
	y_pred = clf.predict(X)
	y_proba = clf.predict_proba(X)[:, 1] if hasattr(clf, "predict_proba") else np.zeros_like(y_pred,
	dtype=float)
	return {
		"precision": 0.0,
		"recall": 0.0,
		"f1_score": 0.0,
		"auc": 0.0
	}
	
	stratify_param = y if min(Counter(y).values()) > 1 else None
	X_train, X_test, y_train, y_test = train_test_split(
	X, y, test_size=0.2, random_state=42, stratify=stratify_param
	)
	
	if isinstance(self.classifier, CascadeForestClassifier):
	clf = CascadeForestClassifier(random_state=42)
	else:
	from copy import deepcopy
	clf = deepcopy(self.classifier)
	
	clf.fit(X_train, y_train)
	y_pred = clf.predict(X_test)
	print("[DEBUG] y_pred распределение:", Counter(y_pred))
	
	if hasattr(clf, "predict_proba"):
	y_proba = clf.predict_proba(X_test)[:, 1]
	print("[DEBUG] y_proba min/max:", y_proba.min(), y_proba.max())
	else:
	y_proba = np.zeros_like(y_pred, dtype=float)
	
	precision = precision_score(y_test, y_pred, zero_division=0)
	recall = recall_score(y_test, y_pred, zero_division=0)
	f1 = f1_score(y_test, y_pred, zero_division=0)
	auc = roc_auc_score(y_test, y_proba) if len(set(y_test)) == 2 else 0.0
	
	return {
		"precision": precision,
		"recall": recall,
		"f1_score": f1,
		"auc": auc
	}
\end{lstlisting}

\begin{lstlisting}[language=Python, caption={{ \texttt{app.py}}}]
	# app.py
	
	import os
	from dotenv import load_dotenv
	import dash
	from dash import dcc, html, dash_table
	from dash.dependencies import Input, Output
	import dash_bootstrap_components as dbc
	import plotly.express as px
	import plotly.graph_objects as go
	import pandas as pd
	import numpy as np
	
	from repository_analysis import GitHubRepoAnalyzer
	from ml_model import CommitRiskModel
	from xgboost import XGBClassifier
	from deepforest import CascadeForestClassifier
	from recommendations import generate_recommendations
	
	# 1. Загрузка настроек
	load_dotenv()
	github_token = os.getenv('GITHUB_TOKEN')
	repos = [r for r in os.getenv("GITHUB_REPOS", "").split(",") if r]
	
	# 2. Сбор и анализ каждого репозитория
	analyses = {}
	for full_name in repos:
	owner, name = full_name.split("/")
	analyzer = GitHubRepoAnalyzer(owner, name, github_token)
	commits = analyzer.analyze_commits()
	analyzer.analyze_and_pr(commits)
	
	# 3. Обучение модели на коммитах
	#model = CommitRiskModel(XGBClassifier(eval_metric="logloss"))
	model = CommitRiskModel(CascadeForestClassifier(random_state=42))
	model.fit(commits)
	metrics = model.evaluate_model(commits)
	
	# 4. Подготовка DataFrame
	df = pd.DataFrame(commits)
	df['Risk_Proba'] = model.predict_proba(commits)
	df['Risk'] = (df['Risk_Proba'] > 0.5).astype(int)
	
	analyses[full_name] = {
		'df': df,
		'model': model,
		'feat_imps': model.feature_importances(),
		'metrics': metrics
	}
	
	# 5. Инициализация Dash-приложения
	app = dash.Dash(__name__, external_stylesheets=[dbc.themes.LUX])
	
	app.layout = dbc.Container(fluid=True, children=[
	html.H1("Мульти-репозиторный анализ коммитов", className="text-center my-4"),
	dcc.Dropdown(
	id="repo-selector",
	options=[{"label": r, "value": r} for r in repos],
	value=repos[0] if repos else None,
	clearable=False,
	style={"width": "60%", "margin": "0 auto 20px auto"}
	),
	html.Div(id="tabs-container")
	])
	
	@app.callback(
	Output("tabs-container", "children"),
	Input("repo-selector", "value")
	)
	def update_tabs(selected_repo):
	if not selected_repo or selected_repo not in analyses:
	return html.Div("Репозиторий не выбран или недоступен")
	
	entry = analyses[selected_repo]
	df = entry['df'].copy()
	feat_imps = entry['feat_imps']
	metrics = entry.get('metrics', {})
	metrics_table = dbc.Table([
	html.Thead(html.Tr([html.Th("Метрика"), html.Th("Значение")])),
	html.Tbody([
	html.Tr([html.Td("Precision"), html.Td(f"{metrics.get('precision', 0):.2f}")]),
	html.Tr([html.Td("Recall"), html.Td(f"{metrics.get('recall', 0):.2f}")]),
	html.Tr([html.Td("F1-score"), html.Td(f"{metrics.get('f1_score', 0):.2f}")]),
	html.Tr([html.Td("ROC-AUC"), html.Td(f"{metrics.get('auc', 0):.2f}")]),
	])
	], bordered=True, striped=True, hover=True, style={"width": "40%", "marginTop": "20px"})
	features = entry['model'].features
	
	# Подстраховки
	if 'author_name' not in df:
	df['author_name'] = 'Unknown'
	if 'has_bug_keyword' not in df:
	df['has_bug_keyword'] = df['message'].str.contains(
	r'\b(fix|bug|error)\b', case=False, regex=True, na=False
	).astype(int)
	
	# 6. Общая информация
	tab_summary = dcc.Tab(label='Общая информация', children=[
	dbc.Row([
	dbc.Col(dcc.Graph(
	figure=px.histogram(
	df, x='lines_added', nbins=30,
	title='Добавленные строки',
	color_discrete_sequence=['#1f77b4']  # синяя
	)
	), md=6),
	dbc.Col(dcc.Graph(
	figure=px.histogram(
	df, x='lines_deleted', nbins=30,
	title='Удалённые строки',
	color_discrete_sequence=['#d62728']  # красная
	)
	), md=6),
	], className="g-4"),
	dbc.Row([
	dbc.Col(dcc.Graph(
	figure=px.histogram(
	df, x='files_changed', nbins=30,
	title='Изменённые файлы',
	color_discrete_sequence=['#2ca02c']  # зелёная
	)
	), md=6),
	dbc.Col(dcc.Graph(
	figure=px.histogram(
	df, x='complexity_score', nbins=30,
	title='Сложность изменений',
	color_discrete_sequence=['#9467bd']  # фиолетовая
	)
	), md=6),
	], className="g-4"),
	])
	
	# 7. Анализ риска
	fi_vals = [feat_imps.get(f, 0) for f in features]
	tab_risk = dcc.Tab(label='Анализ риска', children=[
	metrics_table,
	dbc.Row(dbc.Col(dcc.Graph(
	figure=px.bar(
	x=features, y=fi_vals,
	title='Важность признаков',
	color_discrete_sequence=px.colors.qualitative.Set2
	)
	)), className="g-4"),
	dbc.Row([
	dbc.Col(dcc.Graph(
	figure=px.pie(
	df, names='Risk', title='Рискованные vs обычные',
	color_discrete_sequence=['#2ca02c', '#d62728']
	)
	), md=6),
	dbc.Col(dcc.Graph(
	figure=px.scatter(
	df, x='lines_added', y='complexity_score',
	color='Risk', title='Риск vs Сложность',
	color_discrete_sequence=['#2ca02c', '#d62728']
	)
	), md=6),
	], className="g-4"),
	])
	
	# 8. Авторы
	author_activity = df['author_name'].value_counts().reset_index()
	author_activity.columns = ['author_name', 'commit_count']
	author_risk = df.groupby('author_name')['Risk_Proba'] \
	.mean().reset_index() \
	.sort_values('Risk_Proba', ascending=False)
	
	tab_authors = dcc.Tab(label='Авторы', children=[
	dbc.Row([
	dbc.Col(dcc.Graph(
	figure=px.bar(
	author_activity, x='author_name', y='commit_count',
	title='Активность авторов',
	color_discrete_sequence=px.colors.sequential.Viridis
	)
	), md=6),
	dbc.Col(dcc.Graph(
	figure=px.bar(
	author_risk, x='author_name', y='Risk_Proba',
	title='Средний риск по авторам',
	color_discrete_sequence=px.colors.sequential.Reds
	)
	), md=6),
	], className="g-4"),
	])
	
	# 9. File-Risk Map
	file_df = df.explode('file_list') if 'file_list' in df else pd.DataFrame()
	if not file_df.empty:
	fr = file_df.groupby('file_list').agg(
	change_count=('file_list','size'),
	avg_risk=('Risk_Proba','mean')
	).reset_index()
	else:
	fr = pd.DataFrame(columns=['file_list','change_count','avg_risk'])
	tab_file_risk = dcc.Tab(label='File-Risk Map', children=[
	dbc.Row(dbc.Col(dcc.Graph(
	figure=px.scatter(
	fr, x='change_count', y='avg_risk',
	hover_name='file_list',
	title='Частота изменений vs средний риск',
	color_discrete_sequence=['#17becf']  # бирюзовая
	)
	)), className="g-4")
	])
	
	# 10. Risk Timeline
	df['commit_date'] = pd.to_datetime(df['author_datetime'], errors='coerce').dt.date
	tl = df.sort_values('commit_date').groupby('commit_date').agg(
	daily_risk=('Risk_Proba','mean'),
	warnings=('Risk','sum')
	).reset_index()
	fig_tl = go.Figure([
	go.Scatter(
	x=tl['commit_date'], y=tl['daily_risk'],
	mode='lines+markers', name='Средний риск',
	line=dict(color='#1f77b4')
	),
	go.Bar(
	x=tl['commit_date'], y=tl['warnings'],
	name='Предупреждения', yaxis='y2', opacity=0.6,
	marker_color='#ff7f0e'
	)
	])
	fig_tl.update_layout(
	title='Timeline риска и предупреждений',
	yaxis=dict(title='Средний риск'),
	yaxis2=dict(title='Кол-во предупреждений', overlaying='y', side='right')
	)
	tab_timeline = dcc.Tab(label='Risk Timeline', children=[
	dbc.Row(dbc.Col(dcc.Graph(figure=fig_tl)), className="g-4")
	])
	
	# 11. Code Quality Tabs
	quality_tabs = []
	if {'pylint_warnings','pylint_errors','bandit_issues'} <= set(df.columns):
	quality_tabs.append(dcc.Tab(label='Python Quality', children=[
	dbc.Row([
	dbc.Col(dcc.Graph(
	figure=px.histogram(
	df, x='pylint_warnings',
	title='Pylint Warnings',
	color_discrete_sequence=['#9467bd']
	)
	), md=6),
	dbc.Col(dcc.Graph(
	figure=px.scatter(
	df, x='pylint_errors', y='bandit_issues',
	title='Errors vs Security Issues',
	color_discrete_sequence=['#8c564b']
	)
	), md=6),
	], className="g-4")
	]))
	if {'eslint_warnings','eslint_errors'} <= set(df.columns):
	quality_tabs.append(dcc.Tab(label='JS Quality', children=[
	dbc.Row([
	dbc.Col(dcc.Graph(
	figure=px.histogram(
	df, x='eslint_warnings',
	title='ESLint Warnings',
	color_discrete_sequence=['#e377c2']
	)
	), md=6),
	dbc.Col(dcc.Graph(
	figure=px.scatter(
	df, x='eslint_errors', y='eslint_warnings',
	title='Errors vs Warnings',
	color_discrete_sequence=['#7f7f7f']
	)
	), md=6),
	], className="g-4")
	]))
	if 'checkstyle_issues' in df.columns:
	quality_tabs.append(dcc.Tab(label='Java Quality', children=[
	dbc.Row(dbc.Col(dcc.Graph(
	figure=px.histogram(
	df, x='checkstyle_issues',
	title='Checkstyle Issues',
	color_discrete_sequence=['#bcbd22']
	)
	)), className="g-4")
	]))
	
	# 12. Commits Table
	df['Recommendations'] = df.apply(
	lambda row: generate_recommendations(row, row['Risk_Proba'], {}, feat_imps),
	axis=1
	)
	df['Recommendations_Text'] = df['Recommendations'].apply(lambda recs: "; ".join(recs))
	tab_table = dcc.Tab(label='Commits Table', children=[
	dash_table.DataTable(
	columns=[
	{"name":"SHA","id":"commit"},
	{"name":"Автор","id":"author_name"},
	{"name":"Дата","id":"commit_date"},
	{"name":"Риск","id":"Risk_Proba","type":"numeric","format":{"specifier":".2f"}},
	{"name":"Сообщение","id":"message"},
	{"name":"Рекомендации","id":"Recommendations_Text"},
	],
	data=df[['commit','author_name','commit_date','Risk_Proba','message','Recommendations_Text']]
	.to_dict('records'),
	page_size=10,
	style_cell={'textAlign':'left','whiteSpace':'normal','height':'auto'},
	style_header={'fontWeight':'bold'}
	)
	])
	
	# 13. Календарь активности
	all_dates = pd.date_range(df['commit_date'].min(), df['commit_date'].max(), freq='D')
	heat = pd.DataFrame({'date': all_dates})
	heat['count'] = heat['date'].map(df['commit_date'].value_counts()).fillna(0)
	heat['dow'] = heat['date'].dt.weekday
	heat['week'] = ((heat['date'] - heat['date'].min()).dt.days // 7).astype(int)
	max_w = heat['week'].max() + 1
	mat = np.zeros((7, max_w))
	for _, r in heat.iterrows():
	mat[int(r['dow']), int(r['week'])] = r['count']
	cal_fig = go.Figure(data=go.Heatmap(
	z=mat,
	x=[f'Неделя {i+1}' for i in range(max_w)],
	y=['Пн','Вт','Ср','Чт','Пт','Сб','Вс'],
	colorscale='Greens', hoverongaps=False,
	colorbar=dict(title='Коммитов/день')
	))
	cal_fig.update_layout(xaxis=dict(scaleanchor='y', showgrid=False),
	yaxis=dict(showgrid=False))
	tab_calendar = dcc.Tab(label='Календарь активности', children=[
	dbc.Row(dbc.Col(dcc.Graph(figure=cal_fig)), className="g-4")
	])
	
	tabs = [
	tab_summary,
	tab_risk,
	tab_authors,
	tab_file_risk,
	tab_timeline,
	*quality_tabs,
	tab_table,
	tab_calendar
	]
	return dcc.Tabs(tabs)
	
	if __name__ == '__main__':
	app.run(debug=True)
\end{lstlisting}

\begin{lstlisting}[language=Python, caption={{ \texttt{recommendationds.py}}}]
	# recommendations.py
	from typing import List
	
	__all__ = ['generate_recommendations']
	
	def generate_recommendations(commit: dict,
	risk_proba: float,
	repo_stats: dict,
	feature_importances: dict) -> List[str]:
	
	recommendations: List[str] = []
	
	if risk_proba > 0.8:
	recommendations.append(
	"Очень высокий риск: обязательно провести углублённое код-ревью и расширенное тестирование."
	)
	elif risk_proba > 0.5:
	recommendations.append(
	"Повышенный риск: обратите внимание на качество изменений и добавьте тесты."
	)
	
	msg_len = commit.get('message_length', 0)
	if msg_len < 20:
	recommendations.append(
	"Сообщение слишком короткое: дайте подробное описание изменений."
	)
	elif msg_len > 200:
	recommendations.append(
	"Очень длинное сообщение: разделите описание на ключевые пункты или используйте более лаконичные формулировки."
	)
	
	if commit.get('has_bug_keyword', 0):
	recommendations.append(
	"Выявлен багфикс: убедитесь в наличии регрессионных тестов и обновлении документации."
	)
	
	lines_added = commit.get('lines_added', 0)
	lines_deleted = commit.get('lines_deleted', 0)
	total = lines_added + lines_deleted
	stats_total = repo_stats.get('total_changes', {})
	mean_total = stats_total.get('mean')
	std_total = stats_total.get('std')
	if mean_total and std_total and total > mean_total + 2 * std_total:
	recommendations.append(
	f"Объём изменений ({total}) значительно превышает среднее ({mean_total:.1f}). "
	"Разбейте коммит на более мелкие логические части."
	)
	
	files_changed = commit.get('files_changed', 0)
	stats_files = repo_stats.get('files_changed', {})
	q95_files = stats_files.get('quantile_95')
	if q95_files and files_changed > q95_files:
	recommendations.append(
	f"Затронуто слишком много файлов ({files_changed} > 95% квантиль). Проверьте целостность изменений."
	)
	
	complexity = commit.get('complexity_score', 0)
	stats_complex = repo_stats.get('complexity_score', {})
	q90_complex = stats_complex.get('quantile_90')
	if q90_complex and complexity > q90_complex:
	recommendations.append(
	f"Высокая сложность ({complexity} > 90% квантиль). "
	"Рассмотрите рефакторинг и дополнительное покрытие тестами."
	)
	
	avg_hist = commit.get('avg_file_history', 0)
	stats_hist = repo_stats.get('avg_file_history', {})
	mean_hist = stats_hist.get('mean')
	std_hist = stats_hist.get('std')
	if mean_hist and std_hist and avg_hist > mean_hist + 2 * std_hist:
	recommendations.append(
	"Возможно, стоит разделить функциональность."
	)
	
	interval = commit.get('minutes_since_previous_commit')
	stats_interval = repo_stats.get('commit_interval', {})
	median_int = stats_interval.get('median')
	if interval is not None and median_int:
	if interval < 5:
	recommendations.append(
	"Очень быстрый коммит (<5 минут): убедитесь, что изменения завершены и протестированы."
	)
	if interval > 2 * median_int:
	recommendations.append(
	f"Промежуток {interval:.0f} мин более чем в 2 раза дольше медианы "
	f"({median_int:.0f} мин): проверьте актуальность ветки перед слиянием."
	)
	
	author = commit.get('author_name', 'Unknown')
	author_stats = repo_stats.get('author_stats', {}).get(author, {})
	median_lines_author = author_stats.get('median_lines_added')
	if median_lines_author and lines_added > 2 * median_lines_author:
	recommendations.append(
	f"Автор {author} внёс {lines_added} строк, что более чем в 2 раза превышает "
	f"его медианные {median_lines_author} строк: дополнительная проверка кода."
	)
	
	if not recommendations:
	recommendations.append(
	"Явных аномалий не обнаружено. Рекомендуется стандартное код-ревью и покрытие тестами."
	)
	
	return recommendations
\end{lstlisting}