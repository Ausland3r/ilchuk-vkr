%%% Внесите свои данные - Input your data
%%
%%
\newcommand{\Author}{А. Е.\,Ильчук} % И.О. Фамилия автора 
\newcommand{\AuthorFull}{Ильчук Александр Евгеньевич} % Фамилия Имя Отчество автора
\newcommand{\AuthorFullDat}{Ильчуку Александру Евгеньевичу} % Фамилия Имя Отчество автора в дательном падеже (Кому? Студенту...)
\newcommand{\AuthorFullVin}{Ильчука Александра Евгеньевича} % в винительном падеже (Кого? что?  Програмиста ...)
\newcommand{\AuthorPhone}{+7-931-987-11-60} % номер телефорна автора для оперативной связи  
\newcommand{\Supervisor}{В. А.\,Пархоменко} % И. О. Фамилия научного руководителя
\newcommand{\SupervisorFull}{Пархоменко Владимир Андреевич} % Фамилия Имя Отчество научного руководителя
\newcommand{\SupervisorVin}{В. А.\,Пархоменко} % И. О. Фамилия научного руководителя  в винительном падеже (Кого? что? Руководителя ...)
\newcommand{\SupervisorJob}{Старший преподаватель} %
\newcommand{\SupervisorJobVin}{Старшего преподавателя} % в винительном падеже (Кого? что?  Програмиста ...)
\newcommand{\SupervisorDegree}{ВШ ПИ} %
\newcommand{\SupervisorTitle}{} % 
%%
%%
%Руководитель, утверждающий задание
\newcommand{\Head}{А. В. Щукин} % И. О. Фамилия руководителя подразделения (руководителя ОП)
\newcommand{\HeadVin}{А. В. Щукину}
\newcommand{\HeadDegree}{Доцент}% Только должность:   
%Руководитель %ОП 
%Заведующий % кафедрой
%Директор % Высшей школы
%Зам. директора
\newcommand{\HeadDep}{ВШ ПИ} % заменить на краткую аббревиатуру подразделения или оставить пустым, если утверждает руководитель ОП

%%% Руководитель, принимающий заявление
\newcommand{\HeadAp}{А. В. Щукин} % И. О. Фамилия руководителя подразделения (руководителя ОП)
\newcommand{\HeadApDegree}{РОП Прикладная информатика}% Только должность:   
%Руководитель ОП 
%Заведующий кафедрой
%Директор Высшей школы
\newcommand{\HeadApDep}{} % заменить на краткую аббревиатуру подразделения или оставить пустым, если утверждает руководитель ОП
%%% Консультант по нормоконтролю
\newcommand{\ConsultantNorm}{Е. Е. Андрианова} % И. О. Фамилия консультанта по нормоконтролю. ТОЛЬКО из числа ППС!
\newcommand{\ConsultantNormDegree}{Ст. преподаватель ВШ ПИ} %   
%%% Первый консультант
\newcommand{\ConsultantExtraFull}{Андрианова Екатерина Евгеньевна} % Фамилия Имя Отчетство дополнительного консультанта 
\newcommand{\ConsultantExtra}{Е. Е. Андрианова} % И. О. Фамилия дополнительного консультанта 
\newcommand{\ConsultantExtraDegree}{старший преподаватель} % 
\newcommand{\ConsultantExtraVin}{Е. Е. Андрианову} % И. О. Фамилия дополнительного консультанта в винительном падеже (Кого? что? Руководителя ...)
\newcommand{\ConsultantExtraDegreeVin}{} %  в винительном падеже (Кого? что? Руководителя ...)
%%% Второй консультант
\newcommand{\ConsultantExtraTwoFull}{} % Фамилия Имя Отчетство дополнительного консультанта 
\newcommand{\ConsultantExtraTwo}{} % И. О. Фамилия дополнительного консультанта 
\newcommand{\ConsultantExtraTwoDegree}{} % 
\newcommand{\ConsultantExtraTwoVin}{} % И. О. Фамилия дополнительного консультанта в винительном падеже (Кого? что? Руководителя ...)
\newcommand{\ConsultantExtraTwoDegreeVin}{} %  в винительном падеже (Кого? что? Руководителя ...)
\newcommand{\Reviewer}{} % И. О. Фамилия резензента. Обязателен только для магистров.
\newcommand{\ReviewerDegree}{} % 
%%
%%
\renewcommand{\thesisTitle}{Разработка системы формирования корректирующих и предупредительных действий на основе изменений репозитория кода}
\newcommand{\thesisDegree}{работа бакалавра}% дипломный проект, дипломная работа, магистерская диссертация %c 2020
\newcommand{\thesisTitleEn}{Developing tools for artificially extracting corrective preventive actions from a code repository using analysis tools} %2020
\newcommand{\thesisDeadline}{17.05.2025}
\newcommand{\thesisStartDate}{12.12.2024}
\newcommand{\emptyDate}{<<\uline{\space\space\space\space\space\space}>>\uline{\space\space\space\space\space\space\space\space\space\space\space\space\space\space\space\space}20\uline{\space\space\space\space\space\space}}
\newcommand{\thesisYear}{2025}
%%
%%
\newcommand{\group}{5130903/10301} % заменить вместо N номер группы
\newcommand{\thesisSpecialtyCode}{09.03.03}% код направления подготовки
\newcommand{\thesisSpecialtyTitle}{Прикладная информатика} % наименование направления/специальности
\newcommand{\thesisOPPostfix}{03} % последние цифры кода образовательной программы (после <<_>>)
\newcommand{\thesisOPTitle}{Интеллектуальные инфокоммуникационные технологии}% наименование образовательной программы
%%
%%
\newcommand{\institute}{
	Институт компьютерных наук и кибербезопасности
}%
%%
%%




%%% Задание ключевых слов и аннотации
%%
%%
%% Ключевых слов от 3 до 5 слов или словосочетаний в именительном падеже именительном падеже множественного числа (или в единственном числе, если нет другой формы) по правилам русского языка!!!
%%
%%
\newcommand{\keywordsRu}{Репозиторий, Анализ данных, Метрики, Качество кода} % ВВЕДИТЕ ключевые слова по-русски
%%
%%
\newcommand{\keywordsEn}{Repository, Data analysis, Metrics, Code quality} % ВВЕДИТЕ ключевые слова по-английски
%%
%%
%% Реферат ОТ 1000 ДО 1500 знаков на русский или английский текст
%%
%Реферат должен содержать:
%- предмет, тему, цель ВКР;
%- метод или методологию проведения ВКР:
%- результаты ВКР:
%- область применения результатов ВКР;
%- выводы.

\newcommand{\abstractRu}{} % ВВЕДИТЕ текст аннотации по-русски
%%
%%
\newcommand{\abstractEn}{} % ВВЕДИТЕ текст аннотации по-английски


%%% РАЗДЕЛ ДЛЯ ОФОРМЛЕНИЯ ПРАКТИКИ
%Место прохождения практики
\newcommand{\PracticeType}{Отчет о прохождении % 
	%стационарной производственной (технологической (проектно-технологической)) %
	такой-то % тип и вид ЗАМЕНИТЬ
	практики}

\newcommand{\Workplace}{СПбПУ, ИКНК, ВШПИ} % TODO Rename this variable

% Даты начала/окончания
\newcommand{\PracticeStartDate}{%
	01.09.2024%
	%	22.06.2020
}%
\newcommand{\PracticeEndDate}{%
	25.01.2025%
	%	18.07.2020%
}%
%%

\newcommand{\School}{
	Высшая школа программной инженерии
	%	Высшая школа интеллектуальных систем и~суперкомпьютерных~технологий 
}
\newcommand{\practiceTitle}{Тема практики}


%% ВНИМАНИЕ! Необходимо либо заменить текст аннотации (ключевых слов) на русском и английском, либо удалить там весь текст, иначе в свойства pdf-отчета по практике пойдет шаблонный текст.

%%% Не меняем дальнейшую часть - Do not modify the rest part
%%
%%
%%
%%
\ifnumequal{\value{docType}}{1}{% Если ВКР, то...
	\newcommand{\DocType}{Выпускная квалификационная работа}
	\newcommand{\pdfDocType}{\DocType~(\thesisDegree)} %задаём метаданные pdf файла
	\newcommand{\pdfTitle}{\thesisTitle}
}{% Иначе 
	\newcommand{\DocType}{\PracticeType}
	\newcommand{\pdfDocType}{\DocType} %задаём метаданные pdf файла
	\newcommand{\pdfTitle}{\practiceTitle}
}%
\newcommand{\HeadTitle}{Руководитель ОП}
\newcommand{\HeadApTitle}{\HeadApDegree~\HeadApDep}
\newcommand{\thesisOPCode}{\thesisSpecialtyCode\_\thesisOPPostfix}% код образовательной программы
\newcommand{\thesisSpecialtyCodeAndTitle}{\thesisSpecialtyCode~\thesisSpecialtyTitle}% Код и наименование направления/специальности
\newcommand{\thesisOPCodeAndTitle}{\thesisOPCode~\thesisOPTitle} % код и наименование образовательной программы
%%
%%

%%
%%
%% вспомогательные команды
\newcommand{\firef}[1]{рис.\ref{#1}} %figure reference
\newcommand{\taref}[1]{табл.\ref{#1}}	%table reference
%%
%%

%%% Переопределение именований %%% Не меняем - Do not modify
%\newcommand{\Ministry}{Минобрнауки России} 
\newcommand{\Ministry}{Министерство науки и высшего образования Российской~Федерации} %с 2020
\newcommand{\SPbPU}{Санкт-Петербургский политехнический университет Петра~Великого}
\newcommand{\SPbPUOfficialPrefix}{Федеральное государственное автономное образовательное учреждение высшего образования}
\newcommand{\SPbPUOfficialShort}{ФГАОУ~ВО~<<СПбПУ>>}
%% Пробел между И. О. не допускается.
\renewcommand{\alsoname}{см. также}
\renewcommand{\seename}{см.}
\renewcommand{\headtoname}{вх.}
\renewcommand{\ccname}{исх.}
\renewcommand{\enclname}{вкл.}
\renewcommand{\pagename}{Pages}
\renewcommand{\partname}{Часть}
\renewcommand{\abstractname}{\textbf{Аннотация}}
\newcommand{\abstractnameENG}{\textbf{Annotation}}
\newcommand{\keywords}{\textbf{Ключевые слова}}
\newcommand{\keywordsENG}{\textbf{Keywords}}
\newcommand{\acknowledgements}{\textbf{Благодарности}}
\newcommand{\acknowledgementsENG}{\textbf{Acknowledgements}}
\renewcommand{\contentsname}{Content} % 
%\renewcommand{\contentsname}{Содержание} % (ГОСТ Р 7.0.11-2011, 4)
%\renewcommand{\contentsname}{Оглавление} % (ГОСТ Р 7.0.11-2011, 4)
\renewcommand{\figurename}{Рис.} % Стиль СПбПУ
%\renewcommand{\figurename}{Рисунок} % (ГОСТ Р 7.0.11-2011, 5.3.9)
\renewcommand{\tablename}{Таблица} % (ГОСТ Р 7.0.11-2011, 5.3.10)
%\renewcommand{\indexname}{Предметный указатель}
\renewcommand{\listfigurename}{Список рисунков}
\renewcommand{\listtablename}{Список таблиц}
\renewcommand{\refname}{\fullbibtitle}
\renewcommand{\bibname}{\fullbibtitle}

\newcommand{\chapterEnTitle}{Сhapter title} % <- input the English title here (only once!) 
\newcommand{\chapterRuTitle}{Название главы}          % <- введите 
\newcommand{\sectionEnTitle}{Section title} %<- input subparagraph title in english
\newcommand{\sectionRuTitle}{Название подраздела} % <- введите название подраздела по-русски
\newcommand{\subsectionEnTitle}{Subsection title} % - input subsection title in english
\newcommand{\subsectionRuTitle}{Название параграфа} % <- введите название параграфа по-русски
\newcommand{\subsubsectionEnTitle}{Subsubsection title} % <- input subparagraph title in english
\newcommand{\subsubsectionRuTitle}{Название подпараграфа} % <- введите название подпараграфа по-русски