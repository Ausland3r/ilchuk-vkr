%%% Внесите свои данные - Input your data
%%
%%
\newcommand{\Author}{А. Е.\,Ильчук} % И.О. Фамилия автора 
\newcommand{\AuthorFull}{Ильчук Александр Евгеньевич} % Фамилия Имя Отчество автора
\newcommand{\AuthorFullDat}{Ильчуку Александру Евгеньевичу} % Фамилия Имя Отчество автора в дательном падеже (Кому? Студенту...)
\newcommand{\AuthorFullVin}{Ильчука Александра Евгеньевича} % в винительном падеже (Кого? что?  Програмиста ...)
\newcommand{\AuthorPhone}{+7-931-987-11-60} % номер телефорна автора для оперативной связи  
\newcommand{\Supervisor}{В. А.\,Пархоменко} % И. О. Фамилия научного руководителя
\newcommand{\SupervisorFull}{Пархоменко Владимир Андреевич} % Фамилия Имя Отчество научного руководителя
\newcommand{\SupervisorVin}{В. А.\,Пархоменко} % И. О. Фамилия научного руководителя  в винительном падеже (Кого? что? Руководителя ...)
\newcommand{\SupervisorJob}{Старший преподаватель} %
\newcommand{\SupervisorJobVin}{Старшего преподавателя} % в винительном падеже (Кого? что?  Програмиста ...)
\newcommand{\SupervisorDegree}{ВШ ПИ} %
\newcommand{\SupervisorTitle}{} % 
%%
%%
%Руководитель, утверждающий задание
\newcommand{\Head}{А. В. Щукин} % И. О. Фамилия руководителя подразделения (руководителя ОП)
\newcommand{\HeadVin}{А. В. Щукину}
\newcommand{\HeadDegree}{Доцент}% Только должность:   
%Руководитель %ОП 
%Заведующий % кафедрой
%Директор % Высшей школы
%Зам. директора
\newcommand{\HeadDep}{ВШ ПИ} % заменить на краткую аббревиатуру подразделения или оставить пустым, если утверждает руководитель ОП

%%% Руководитель, принимающий заявление
\newcommand{\HeadAp}{А. В. Щукин} % И. О. Фамилия руководителя подразделения (руководителя ОП)
\newcommand{\HeadApDegree}{РОП Прикладная информатика}% Только должность:   
%Руководитель ОП 
%Заведующий кафедрой
%Директор Высшей школы
\newcommand{\HeadApDep}{} % заменить на краткую аббревиатуру подразделения или оставить пустым, если утверждает руководитель ОП
%%% Консультант по нормоконтролю
\newcommand{\ConsultantNorm}{Е. Е. Андрианова} % И. О. Фамилия консультанта по нормоконтролю. ТОЛЬКО из числа ППС!
\newcommand{\ConsultantNormDegree}{Ст. преподаватель ВШ ПИ} %   
%%% Первый консультант
\newcommand{\ConsultantExtraFull}{Андрианова Екатерина Евгеньевна} % Фамилия Имя Отчетство дополнительного консультанта 
\newcommand{\ConsultantExtra}{Е. Е. Андрианова} % И. О. Фамилия дополнительного консультанта 
\newcommand{\ConsultantExtraDegree}{старший преподаватель} % 
\newcommand{\ConsultantExtraVin}{Е. Е. Андрианову} % И. О. Фамилия дополнительного консультанта в винительном падеже (Кого? что? Руководителя ...)
\newcommand{\ConsultantExtraDegreeVin}{} %  в винительном падеже (Кого? что? Руководителя ...)
%%% Второй консультант
\newcommand{\ConsultantExtraTwoFull}{} % Фамилия Имя Отчетство дополнительного консультанта 
\newcommand{\ConsultantExtraTwo}{} % И. О. Фамилия дополнительного консультанта 
\newcommand{\ConsultantExtraTwoDegree}{} % 
\newcommand{\ConsultantExtraTwoVin}{} % И. О. Фамилия дополнительного консультанта в винительном падеже (Кого? что? Руководителя ...)
\newcommand{\ConsultantExtraTwoDegreeVin}{} %  в винительном падеже (Кого? что? Руководителя ...)
\newcommand{\Reviewer}{} % И. О. Фамилия резензента. Обязателен только для магистров.
\newcommand{\ReviewerDegree}{} % 
%%
%%
\renewcommand{\thesisTitle}{Разработка системы формирования корректирующих и предупредительных действий на основе изменений репозитория кода}
\newcommand{\thesisDegree}{работа бакалавра}% дипломный проект, дипломная работа, магистерская диссертация %c 2020
\newcommand{\thesisTitleEn}{Developing tools for artificially extracting corrective preventive actions from a code repository using analysis tools} %2020
\newcommand{\thesisDeadline}{17.05.2025}
\newcommand{\thesisStartDate}{12.12.2024}
\newcommand{\emptyDate}{<<\uline{\space\space\space\space\space\space}>>\uline{\space\space\space\space\space\space\space\space\space\space\space\space\space\space\space\space}20\uline{\space\space\space\space\space\space}}
\newcommand{\thesisYear}{2025}
%%
%%
\newcommand{\group}{5130903/10301} % заменить вместо N номер группы
\newcommand{\thesisSpecialtyCode}{09.03.03}% код направления подготовки
\newcommand{\thesisSpecialtyTitle}{Прикладная информатика} % наименование направления/специальности
\newcommand{\thesisOPPostfix}{03} % последние цифры кода образовательной программы (после <<_>>)
\newcommand{\thesisOPTitle}{Интеллектуальные инфокоммуникационные технологии}% наименование образовательной программы
%%
%%
\newcommand{\institute}{
	Институт компьютерных наук и кибербезопасности
}%
%%
%%




%%% Задание ключевых слов и аннотации
%%
%%
%% Ключевых слов от 3 до 5 слов или словосочетаний в именительном падеже именительном падеже множественного числа (или в единственном числе, если нет другой формы) по правилам русского языка!!!
%%
%%
\newcommand{\keywordsRu}{Репозиторий, Анализ данных, Метрики, Качество кода} % ВВЕДИТЕ ключевые слова по-русски
%%
%%
\newcommand{\keywordsEn}{Repository, Data analysis, Metrics, Code quality} % ВВЕДИТЕ ключевые слова по-английски
%%
%%
%% Реферат ОТ 1000 ДО 1500 знаков на русский или английский текст
%%
%Реферат должен содержать:
%- предмет, тему, цель ВКР;
%- метод или методологию проведения ВКР:
%- результаты ВКР:
%- область применения результатов ВКР;
%- выводы.

\newcommand{\abstractRu}{Объект исследования – процесс автоматизированного анализа истории коммитов в Git-репозиториях с целью выявления «рисковых» изменений и генерации корректирующих и предупреждающих рекомендаций (CAPA). 
	
Предмет исследования – методы извлечения и обработки метрик коммитов, алгоритмы кластеризации и классификации «рисковых» изменений, а также подходы к формированию рекомендаций CAPA и их визуализации.
	
	Цель работы – разработать модульную систему, способную автоматически собирать данные о коммитах из удалённых и локальных репозиториев, проводить статический анализ, обучать модель машинного обучения для предсказания риска коммита и формировать для каждого потенциально «рискового» изменения набор рекомендаций CAPA.
	
	В работе выполнены следующие основные этапы:
\begin{enumerate}
	\item Сбор и подготовка данных. Реализован класс \texttt{GitHubRepoAnalyzer}, который через GitHub API или локальный клон последовательно извлекает историю коммитов, собирает базовые метрики (число добавленных/удалённых строк, число изменённых файлов, время между коммитами) и дополняет их результатами статических анализаторов (\texttt{pylint}, \texttt{Bandit}, \texttt{ESLint}, \texttt{Checkstyle}).
	\item Генерация псевдометок и классификация. Для автоматической разметки выполнена кластеризация методом \texttt{KMeans}, разделяющая коммиты на «нормальные» и «аномальные» по признакам объёма и качества кода. На основе полученных меток обучается модель, обёрнутая в универсальный класс \texttt{CommitRiskModel}, позволяющий тестировать различные классификаторы (\texttt{LogisticRegression}, \texttt{RandomForest}, \texttt{SVM}, \texttt{LightGBM}, \texttt{DeepForest} и др.) без изменения кода. Модель предсказывает вероятность «рискового» коммита и вычисляет важность признаков.
	\item Формирование рекомендаций CAPA. На основании вероятности риска и значений ключевых признаков (\texttt{lines\_added}, \texttt{lines\_deleted}, число статических предупреждений) формируется набор текстовых рекомендаций. При превышении пороговых значений формируются предупреждающие сообщения и рекомендации по доработке (добавление тестов, рефакторинг, дополнительная проверка кода). Система автоматически создаёт Pull Request с файлом в формате Markdown, содержащим CAPA-рекомендации.
	\item Визуализация. Разработан интерактивный дашборд на \texttt{Dash} + \texttt{Plotly}, включающий разделы:
	\begin{itemize}
		\item распределение метрик коммитов (\texttt{lines\_added}, \texttt{lines\_deleted}, число файлов),
		\item анализ риска (распределение коммитов по вероятностям, важность признаков),
		\item активность авторов (графики среднего риска по каждому разработчику),
		\item карта риска файлов (\textit{File-Risk Map}) и временная шкала (\textit{Risk Timeline}),
		\item таблица коммитов с рекомендациями CAPA и возможностью фильтрации по проекту, автору и дате.
	\end{itemize}
\end{enumerate}
	
	В качестве тестовых примеров использовались несколько реальных учебных и рабочих репозиториев. Проведено сравнительное тестирование классификаторов на основе метрик Precision, Recall, F1-score и ROC-AUC; для каждого проекта определены оптимальные алгоритмы.  
	
	Система автоматизирует полный цикл анализа коммитов и выдачи рекомендаций CAPA, сокращая время обратной связи, улучшая качество кода и снижая число «рисковых» изменений.}% ВВЕДИТЕ текст аннотации по-русски
%%
%%
\newcommand{\abstractEn}{
	The object of research is the process of automated analysis of commit history in Git repositories to identify “risky” changes and generate corrective and preventive (CAPA) recommendations.
	
	The subject of research includes methods for extracting and processing commit metrics, clustering and classification algorithms for “risky” changes, as well as approaches to forming CAPA recommendations and their visualization.
	
	The aim of this work is to develop a modular system capable of automatically collecting commit data from remote and local repositories, performing static analysis, training a machine learning model to predict commit risk, and generating a set of CAPA recommendations for each potentially “risky” change.
	
	The main stages completed in this work are as follows:
	\begin{enumerate}
		\item Data Collection and Preparation. The \texttt{GitHubRepoAnalyzer} class was implemented, which uses the GitHub API or a local clone to sequentially retrieve commit history, gather basic metrics (number of lines added/deleted, number of files changed, time between commits), and augment them with results from static analyzers (\texttt{pylint}, \texttt{Bandit}, \texttt{ESLint}, \texttt{Checkstyle}).
		\item Pseudo-Label Generation and Classification. For automated labeling, clustering with the \texttt{KMeans} algorithm was performed, splitting commits into “normal” and “anomalous” based on code volume and quality metrics. On the basis of these pseudo-labels, a model is trained using the universal \texttt{CommitRiskModel} class, which allows testing various classifiers (\texttt{LogisticRegression}, \texttt{RandomForest}, \texttt{SVM}, \texttt{LightGBM}, \texttt{DeepForest}, etc.) without modifying the code. The model predicts the probability of a commit being “risky” and calculates feature importances.
		\item CAPA Recommendation Generation. Based on the predicted risk probability and key feature values (\texttt{lines\_added}, \texttt{lines\_deleted}, number of static warnings), a set of textual recommendations is formed. When threshold values are exceeded, warning messages and improvement suggestions are generated (adding tests, refactoring, additional code review). The system automatically creates a Pull Request with a Markdown file containing the CAPA recommendations.
		\item Visualization. An interactive dashboard was developed using \texttt{Dash} + \texttt{Plotly}, which includes:
		\begin{itemize}
			\item distribution of commit metrics (\texttt{lines\_added}, \texttt{lines\_deleted}, number of files),
			\item risk analysis (distribution of commits by probability, feature importance),
			\item author activity (average risk charts per developer),
			\item file risk map and risk timeline,
			\item a commit table with CAPA recommendations and filtering options by project, author, and date.
		\end{itemize}
	\end{enumerate}
	
	Several real educational and work repositories were used as test examples. Comparative testing of classifiers was conducted using Precision, Recall, F1-score, and ROC-AUC metrics; optimal algorithms were identified for each project.
	
	The system automates the full cycle of commit analysis and CAPA recommendation generation, reducing feedback time, improving code quality, and decreasing the number of “risky” changes.
} % ВВЕДИТЕ текст аннотации по-английски


%%% РАЗДЕЛ ДЛЯ ОФОРМЛЕНИЯ ПРАКТИКИ
%Место прохождения практики
\newcommand{\PracticeType}{Отчет о прохождении % 
	%стационарной производственной (технологической (проектно-технологической)) %
	такой-то % тип и вид ЗАМЕНИТЬ
	практики}

\newcommand{\Workplace}{СПбПУ, ИКНК, ВШПИ} % TODO Rename this variable

% Даты начала/окончания
\newcommand{\PracticeStartDate}{%
	01.09.2024%
	%	22.06.2020
}%
\newcommand{\PracticeEndDate}{%
	25.01.2025%
	%	18.07.2020%
}%
%%

\newcommand{\School}{
	Высшая школа программной инженерии
	%	Высшая школа интеллектуальных систем и~суперкомпьютерных~технологий 
}
\newcommand{\practiceTitle}{Тема практики}


%% ВНИМАНИЕ! Необходимо либо заменить текст аннотации (ключевых слов) на русском и английском, либо удалить там весь текст, иначе в свойства pdf-отчета по практике пойдет шаблонный текст.

%%% Не меняем дальнейшую часть - Do not modify the rest part
%%
%%
%%
%%
\ifnumequal{\value{docType}}{1}{% Если ВКР, то...
	\newcommand{\DocType}{Выпускная квалификационная работа}
	\newcommand{\pdfDocType}{\DocType~(\thesisDegree)} %задаём метаданные pdf файла
	\newcommand{\pdfTitle}{\thesisTitle}
}{% Иначе 
	\newcommand{\DocType}{\PracticeType}
	\newcommand{\pdfDocType}{\DocType} %задаём метаданные pdf файла
	\newcommand{\pdfTitle}{\practiceTitle}
}%
\newcommand{\HeadTitle}{Руководитель ОП}
\newcommand{\HeadApTitle}{\HeadApDegree~\HeadApDep}
\newcommand{\thesisOPCode}{\thesisSpecialtyCode\_\thesisOPPostfix}% код образовательной программы
\newcommand{\thesisSpecialtyCodeAndTitle}{\thesisSpecialtyCode~\thesisSpecialtyTitle}% Код и наименование направления/специальности
\newcommand{\thesisOPCodeAndTitle}{\thesisOPCode~\thesisOPTitle} % код и наименование образовательной программы
%%
%%

%%
%%
%% вспомогательные команды
\newcommand{\firef}[1]{рис.\ref{#1}} %figure reference
\newcommand{\taref}[1]{табл.\ref{#1}}	%table reference
%%
%%

%%% Переопределение именований %%% Не меняем - Do not modify
%\newcommand{\Ministry}{Минобрнауки России} 
\newcommand{\Ministry}{Министерство науки и высшего образования Российской~Федерации} %с 2020
\newcommand{\SPbPU}{Санкт-Петербургский политехнический университет Петра~Великого}
\newcommand{\SPbPUOfficialPrefix}{Федеральное государственное автономное образовательное учреждение высшего образования}
\newcommand{\SPbPUOfficialShort}{ФГАОУ~ВО~<<СПбПУ>>}
%% Пробел между И. О. не допускается.
\renewcommand{\alsoname}{см. также}
\renewcommand{\seename}{см.}
\renewcommand{\headtoname}{вх.}
\renewcommand{\ccname}{исх.}
\renewcommand{\enclname}{вкл.}
\renewcommand{\pagename}{Pages}
\renewcommand{\partname}{Часть}
\renewcommand{\abstractname}{\textbf{Аннотация}}
\newcommand{\abstractnameENG}{\textbf{Annotation}}
\newcommand{\keywords}{\textbf{Ключевые слова}}
\newcommand{\keywordsENG}{\textbf{Keywords}}
\newcommand{\acknowledgements}{\textbf{Благодарности}}
\newcommand{\acknowledgementsENG}{\textbf{Acknowledgements}}
\renewcommand{\contentsname}{Content} % 
%\renewcommand{\contentsname}{Содержание} % (ГОСТ Р 7.0.11-2011, 4)
%\renewcommand{\contentsname}{Оглавление} % (ГОСТ Р 7.0.11-2011, 4)
\renewcommand{\figurename}{Рис.} % Стиль СПбПУ
%\renewcommand{\figurename}{Рисунок} % (ГОСТ Р 7.0.11-2011, 5.3.9)
\renewcommand{\tablename}{Таблица} % (ГОСТ Р 7.0.11-2011, 5.3.10)
%\renewcommand{\indexname}{Предметный указатель}
\renewcommand{\listfigurename}{Список рисунков}
\renewcommand{\listtablename}{Список таблиц}
\renewcommand{\refname}{\fullbibtitle}
\renewcommand{\bibname}{\fullbibtitle}

\newcommand{\chapterEnTitle}{Сhapter title} % <- input the English title here (only once!) 
\newcommand{\chapterRuTitle}{Название главы}          % <- введите 
\newcommand{\sectionEnTitle}{Section title} %<- input subparagraph title in english
\newcommand{\sectionRuTitle}{Название подраздела} % <- введите название подраздела по-русски
\newcommand{\subsectionEnTitle}{Subsection title} % - input subsection title in english
\newcommand{\subsectionRuTitle}{Название параграфа} % <- введите название параграфа по-русски
\newcommand{\subsubsectionEnTitle}{Subsubsection title} % <- input subparagraph title in english
\newcommand{\subsubsectionRuTitle}{Название подпараграфа} % <- введите название подпараграфа по-русски